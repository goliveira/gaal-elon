\documentclass[a4paper,11pt]{article}
\usepackage[brazilian]{babel}
\usepackage[T1]{fontenc}
\input{~/Templates/goliveira.tex}

\title{GAAL -- Seção 1 -- Exercício 2}
\author{\empty}
\date{\empty}
\newcommand\onlyinsubfileone\maketitle

\begin{document}

\onlyinsubfileone

\textbf{E2.S1.}
Sejam $a < x < b$ respectivamente as coordenadas dos pontos $A$, $X$ e $B$ do eixo $E$.
Diz-se que o ponto $X$ divide o segmento $AB$ em \emph{média e extrema razão} quando se tem
\[
  \frac{d(A,X)}{d(A,B)} = \frac{d(X,B)}{d(A,X)}.
\]
(O quociente $d(A,X)/d(A,B)$ é chamado \emph{razão áurea}.)
Supondo que $X$ divide o segmento de reta $AB$ em média e extrema razão, calcule $x$ em função de $a$ e $b$.

\vspace{\baselineskip}

\emph{Solução.}
Em coordenadas, a condição dada corresponde a
\[
  \frac{|a-x|}{|a-b|} = \frac{|x-b|}{|a-x|}.
\]
Como $a < x < b$, essa igualdade é equivalente a
\[
  \frac{x-a}{b-a} = \frac{b-x}{x-a},
\]
ou seja,
\[
  x^2 + (b-3a) x + (a^2-b^2+ab) = 0.
\]
O discriminante dessa equação é $\Delta = 5(b-a)^2$.
Portanto as raízes são
\[
  x_{\pm} = \frac{1}{2} (3a - b \pm \sqrt{5}(b-a)).
\]
Usando a condição $a < x < b$, obtemos que $a < x_+ < b$ e $x_- < a$.
Logo a única raiz no intervalo $[a,b]$ é $x_+$.
Portanto o ponto $X$ procurado tem coordenada
\[
  x = \frac{1}{2} ( (3-\sqrt{5}) a + (\sqrt{5}-1) b ).
\]

\end{document}
