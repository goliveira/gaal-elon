\documentclass[a4paper,11pt]{article}
\usepackage[brazilian]{babel}
\usepackage[utf8]{inputenc}
\usepackage[T1]{fontenc}
\usepackage{amsmath}
\usepackage{amssymb}
\usepackage{amsthm}
\usepackage{indentfirst}
\usepackage{enumerate}

\newcommand{\R}{\mathbb{R}}

\title{Exercícios Resolvidos do Livro\\Geometria Analítica e Álgebra Linear\\de Elon Lages Lima\\(Segunda Edição--Oitava Impressão)}
\author{Gustavo de Oliveira}

\begin{document}

\maketitle

\section*{Seção 1 -- Coordenadas na Reta}

\textbf{1.}
Sejam $a < b$ respectivamente as coordenadas dos pontos $A$ e $B$ sobre o eixo $E$.
Determine as coordenadas dos pontos $X_1, \dots, X_{n-1}$ que dividem o segmento $AB$ em $n$ partes iguais.

\vspace{\baselineskip}

\emph{Solução.}
Para $j \in \{1, \dots, n-1 \}$, observamos que $d(X_j,A) = j d(A,B)/n$.
Seja $x_j$ a coordenada do ponto $X_j$.
Então $|x_j - a| = j|a-b|/n$, ou seja, $x_j - a = j(b-a)/n$, pois $x_j > a$ e $b > a$.
Portanto $x_j = a + j(b-a)/n$, ou ainda $x_j = (1-j/n) + (j/n)b$ para $j \in \{1, \dots, n-1 \}$.

\vspace{\baselineskip}

\textbf{2.}
Sejam $a$, $x$ e $b$, com $a < x < b$, as coordenadas dos pontos $A$, $X$ e $B$ sobre um eixo $\mathcal{E}$, respectivamente.
Dizemos que o ponto $X$ divide o segmento de reta $AB$ em \emph{média e extrema razão} se $X$ satisfaz
\[
  \frac{d(A,X)}{d(A,B)} = \frac{d(X,B)}{d(A,X)}.
\]
O quociente $d(A,X)/d(A,B)$ é chamado \emph{razão áurea}.
Supondo que $X$ divide o segmento de reta $AB$ em média e extrema razão, calcule $x$ em função de $a$ e $b$.

\vspace{\baselineskip}

\emph{Solução.}
Usando coordenadas, a condição se escreve
\[
  \frac{|a-x|}{|a-b|} = \frac{|x-b|}{|a-x|}.
\]
Como $a < x < b$, essa expressão é equivalente a
\[
  \frac{x-a}{b-a} = \frac{b-x}{x-a},
\]
ou seja,
\[
  x^2 + (b-3a) x + (a^2-b^2+ab) = 0.
\]
O discriminante dessa equação é $\Delta = 5(b-a)^2$.
Portanto as raízes da equação são
\[
  x_{\pm} = \frac{1}{2} (3a - b \pm \sqrt{5}(b-a)).
\]
Usando a condição $a < x < b$, obtemos que $a < x_+ < b$ e $x_- < a$.
Logo a única raiz no intervalo $[a,b]$ é $x_+$.
Portanto o ponto $X$ procurado tem coordenada
\[
  x = \frac{1}{2} ( (3-\sqrt{5}) a + (\sqrt{5}-1) b ).
\]

\vspace{\baselineskip}

\textbf{3.}
Seja $O\,$ a origem de um eixo $\mathcal{E}$, e seja $A\,$ o ponto desse eixo cuja coordenada é igual a $1$.
Qual é a coordenada do ponto $X$ que divide o segmento de reta $OA$ em média e extrema razão?
No Exercício 2, calcule a \emph{razão áurea} $d(O,X)/d(O,A)$ (veja o exercício anterior).

\vspace{\baselineskip}

\emph{Solução.}
O ponto $X$ tem coordenada
\[
  x = \frac{1}{2} ( (3-\sqrt{5}) 0 + (\sqrt{5}-1) 1 ) = \frac{\sqrt{5}-1}{2}.
\]
A razão áurea é
\[
  \frac{d(O,X)}{d(O,A)} = \frac{\sqrt{5}-1}{2}.
\]

\vspace{\baselineskip}

\textbf{6.}
Sejam $a < b < c$ respectivamente as coordenadas dos pontos $A$, $B$ e $C$ situados sobre um eixo.
Sabendo que $a = 17$, $c = 32$ e
\[
  \frac{d(A,B)}{d(A,C)} = \frac{2}{3},
\]
qual é o valor de $b$?

\vspace{\baselineskip}

\emph{Solução.}
Usando a fórmula $d(X,Y) = |x-y|$, temos que
\[
  \frac{|17-b|}{|17-32|} = \frac{3}{2}.
\]
Como $b > 17$ e $32 > 17$, essa equação é equivalente a
\[
  \frac{b-17}{32-17} = \frac{2}{3},
\]
ou seja,
\[
  b = \frac{2}{3} 15 + 17 = 27.
\]

\vspace{\baselineskip}

\textbf{7.}
Qual seria a resposta do exercício anterior se soubéssemos apenas que $a < c$?

\vspace{\baselineskip}

\emph{Solução.}
Se soubéssemos apenas que $a < c$, poderia ocorrer que $B$ estivesse à direita de $A$ ou $B$ estivesse à esquerda de $A$.
Logo, além da caso considerado no item (a), teríamos o caso em que $b < 17$ de modo que $|17-b| = 17-b$.
Dessa forma teríamos
\[
  \frac{17-b}{32-17} = \frac{2}{3},
\]
ou seja,
\[
  -b = \frac{2}{3} 15 - 17 = -7,
\]
o que implica em $b = 7$.

\vspace{\baselineskip}

\textbf{9.}
Seja $f : \R \to \R$ uma função tal que $|f(x)-f(y)| = |x-y|$ para quaisquer $x,y \in \R$.
\begin{enumerate}[(i)]
  \item
    Pondo $f(0) = a$, defina a função $g : \R \to \R$ assim: $g(x) = f(x) - a$.
    Prove então que $|g(x)| = |x|$ para todo $x \in \R$.
    Em particular, $g(1) = 1$ ou $g(1) = -1$.
    Também $(g(x))^2 = x^2$.
  \item
    Use a identidade $xy = \frac{1}{2}[ x^2 + y^2 - (x-y)^2 ]$ para mostrar que $xy = g(x) g(y)$.
  \item
    Se $g(1) = 1$, mostre que $g(x) = x$ para todo $x \in \R$.
    Se $g(1) = -1$, mostre que $g(x) = -x$ para todo $x$.
  \item
    Conclua que $f(x) = x + a$ para todo $x \in \R$ ou então $f(x) = -x + a$ para todo $x$.
\end{enumerate}

\vspace{\baselineskip}

\emph{Solução.}
(i)
Observamos que $g(x) = f(x) - a = f(x) - f(0)$.
Logo $|g(x)| = |f(x)-f(0)| = |x-0| = |x|$ para todo $x \in \R$.
Sendo assim, $|g(1)| = 1$, o que implica $g(1) = 1$ ou $g(1) = -1$.
Temos também que $g(x)^2 = |g(x)|^2 = |x|^2 = x^2$.

(ii)
Usando as definições e propriedades, calculamos
\begin{align*}
  xy & = \frac{1}{2} [x^2 + y^2 - (x-y)^2 ] \\
  & = \frac{1}{2} [ x^2 + y^2 - |x-y|^2 ] \\
  & = \frac{1}{2} [ g(x)^2 + g(y)^2 - |f(x) - f(y)|^2 ] \\
  & = \frac{1}{2} [ g(x)^2 + g(y)^2 - |f(x) - a - f(y) + a|^2 ] \\
  & = \frac{1}{2} [ g(x)^2 + g(y)^2 - |g(x) - g(y)|^2 ] \\
  & = \frac{1}{2} [ g(x)^2 + g(y)^2 - (g(x) - g(y))^2 ] \\
  & = g(x)g(y).
\end{align*}

(iii)
Se $g(1)=1$, então $x = x(1) = g(x)g(1) = g(x)$ para todo $x \in \R$.
Se $g(1)=-1$, então $x = x(1) = g(x)g(1) = g(x)(-1) = -g(x)$ para todo $x \in \R$, ou seja, $g(x) = -x$ para todo $x$.

(iv)
Observamos que $f(x) = g(x) + a$.
Pela parte $(i)$, temos $g(1) = 1$ ou $g(1) = -1$.
Usando (iii), isso implica $g(x)=x$ ou $g(x) = -x$ para todo $x$, respectivamente.
Portanto $f(x) = x + a$ ou $f(x) = -x + a$ para todo $x \in \R$.

\section*{Seção 5 -- Escolhendo o Sistema de Coordenadas}

\textbf{2.}
O triângulo $ABC$ é equilátero e cada lado mede $l$.
Num sistema de coordenadas em que a origem é equidistante de $A$, $B$ e $C$, e o ponto $C$ está sobre o eixo $OY$, quais são as coordenadas dos três vértices?

\vspace{\baselineskip}

\emph{Solução.}
Tomando o sistema de coordenadas $OXY$ da forma sugerida, obtemos os triângulo $ABO$, $BCO$ e $CAO$ com $O = (0,0)$ e $A$, $B$ e $C$ da forma $A = (-a,-b)$, $B = (a,-b)$ e $C = (0,c)$.
Esses três triângulo são isóceles e congruentes.
Por exemplo temos $ABO$ com $\hat{A} = \hat{B} = 30^0$ e $\hat{O} = 120^o$.
Logo
\[
  d(A,O) \cos 30^o = l/2,
\]
ou seja, $d(A,O) \sqrt{3}/2 = l/2$, ou seja, $d(A,O) = l/\sqrt{3}$.
Seja $M$ o ponto médio de $AB$.
Então
\[
  d(M,O) = d(A,O) \sin 30^o,
\]
ou seja, $d(M,O) = (l/\sqrt{3})(1/2) = l/(2\sqrt{3})$.
Logo
\begin{align*}
  a & = l/2, \\
  b & = d(M,O) = l/(2\sqrt{3}), \\
  c & = d(O,C) = d(O,A) = l/\sqrt{3}.
\end{align*}
Portanto
\[
  A = \left(-\frac{l}{2}, -\frac{l}{2\sqrt{3}} \right), \qquad B = \left( \frac{l}{2}, \frac{l}{2\sqrt{3}} \right), \qquad C = \left(0, \frac{l}{\sqrt{3}} \right).
\]

\section*{Seção 7 -- As Equações da Reta}

\textbf{10.}
Sejam $A = (1,2)$, $B = (2,4)$ e $C = (3,1)$.
Ache as equações da mediana e da altura do triângulo $ABC$ que partem do vértice $A$.

\vspace{\baselineskip}

\emph{Solução.}
A mediana que parte do vértice $A$ é o segmento $AM$ onde $M$ é o ponto médio do lado $BC$.
Calculando $M$, obtemos $M = (5/2,3/2)$.
A reta que passa por $A$ e $M$ tem inclinação $[(3/2)-2]/[(5/2)-1] = -1/3$, ou seja, a equação dessa reta é da forma $y = -(1/3)x + b$.
Calculando $b$, obtemos $b = 7/3$.
Logo a equação da reta $AM$ é $y = -(1/3)x + 7/3$.

A reta que contém a altura do vértice $A$ é a reta perpendicular ao lado $BC$ passando por $A$.
O lado $BC$ é paralelo a $OC'$ com $C' = (1,-5)$.
Logo a equação da reta tem a forma $x - 5y = c$.
Como a reta passa por $A$, devemos ter $1-5(2) = c$, ou seja, $c = -9$.
Portanto a equação da reta é $x - 5y = -9$.

\vspace{\baselineskip}

\textbf{22.}
Qual é a distância entre as retas paralelas $x - 3y = 4$ e $2x - 6y = 1$?

\vspace{\baselineskip}

\emph{Solução.}
Sejam $r$ e $s$ as retas definidas por $x - 3y = 4$ e $2x - 6y = 1$, respectivamente.
Seja $t$ a reta perpendicular a $r$ (e portanto a $s$) que passa pela origem.
Uma equação para $t$ é $3x + y = 0$.
Calculamos $P = r \cap t$, ou seja, resolvemos o sistema $x-3y = 4$, $3x + y = 0$.
A solução desse sistema é $x = 2/5$ e $y = -6/5$.
Portanto $P = (2/5, -6/5)$.
Calculamos $Q = s \cap t$, ou seja, resolvemos o sistema $2x - 6y = 1$, $3x + y = 0$.
A solução desse sistema é $x = 1/20$ e $y = -3/20$.
Portanto $Q = (1/20, -3/20)$.
Observamos que $d(r,s) = d(P,Q)$.
Calculando $d(P,Q)$, concluímos que
$d(r,s) = 7/(2 \sqrt{10})$.

\section*{Seção 12 -- Equação da Circunferência}

\textbf{4.}
Qual é a equação da circunferência que passa pelos pontos $A = (1,2)$, $B = (3,4)$ e tem o centro sobre o eixo $OY$?

\vspace{\baselineskip}

\emph{Solução.}
Como o centro da circunferência está sobre o eixo $OY$, a equação da circunferência tem a forma $x^2 + (y-b)^2 = r^2$.
Como $A$ e $B$ pertencem à circunferência, devemos ter
\begin{align*}
  1 + (2 - b)^2 & = r^2 \\
  9 + (4 - b)^2 & = r^2.
\end{align*}
Resolvendo esse sistema, obtemos $b = 5$ e $r = \sqrt{10}$.
Portanto a equação da circunferência é $x^2 + (y - 5)^2 = 10$.

\vspace{\baselineskip}

\textbf{5.}
Escreva a equação da circunferência que tem centro no ponto $P = (2,5)$ e é tangente à reta $y = 3x + 1$.

\vspace{\baselineskip}

\emph{Solução.}
A equação da circunferência tem a forma $(x-2)^2 + (y-5)^2 = r^2$.
Para que a circunferência seja tangente à reta, o sistema
\begin{align*}
  & (x-2)^2 + (y-5)^2 = r^2 \\
  & y = 3x + 1
\end{align*}
deve possuir apenas uma solução.
Substituindo a segunda equação na primeira, obtemos $10x^2 - 28x + 20-r^2 = 0$.
Essa equação em $x$ tem apenas uma solução se e somente se $28^2 - 800 + 40r^2 = 0$, ou seja, $r = \sqrt{2/5}$.
Portanto a equação da circunferência é $(x-2)^2 + (y-5)^2 = 2/5$.

\section*{Seção 14 -- Vetores no Plano}

\textbf{2.}
Prove geometricamente que um quadrilátero é um paralelogramo se, e somente se, suas diagonais se cortam mutuamente ao meio.

\vspace{\baselineskip}

\emph{Solução.}
($\Rightarrow$)
Suponha que o quadrilátero $ABCD$ é um paralelogramo.
O paralelogramo é formado por dois pares de lados.
Em cada par de lados, os lados são paralelos e têm o mesmo comprimento.
Portanto $\overrightarrow{AD} = \overrightarrow{BC}$ e $\overrightarrow{AB} = \overrightarrow{DC}$.
Seja $P$ o ponto médio de $DB$, e seja $Q$ o ponto médio de $AC$.
Vamos provar que $Q = P$.
Escolhemos um sistema de coordenadas $OXY$ de modo que $A = (0,0)$, $B = (b,0)$ e $D = (c, d)$.
Logo $\overrightarrow{AD} = (c,d)$ e $C = B + \overrightarrow{AD} = (b + c, d)$.
Calculando os pontos $P$ e $Q$, obtemos
\begin{align*}
  P & = \left( \frac{c+b}{2}, \frac{d+0}{2} \right) = \left( \frac{b+c}{2}, \frac{d}{2} \right), \\
  Q & = \left( \frac{b + c + 0}{2}, \frac{d+0}{2} \right) = \left( \frac{b+c}{2}, \frac{d}{2} \right).
\end{align*}
Portanto $P = Q$.

($\Leftarrow$)
Seja $P$ o ponto médio de $DB$, e seja $Q$ o ponto médio de $AC$.
Suponha que as diagonais do paralelogramo se cortam mutuamente ao meio, ou seja, suponha que $P = Q$.
Escolhemos um sistema de coordenadas $OXY$ de modo que $A = (0,0)$, $B = (b,0)$ e $D = (c, d)$.
Temos então $\overrightarrow{AD} = (c,d)$ e $\overrightarrow{AB} = (b,0)$.
Escrevemos $C = (x,y)$.
Vamos determinar $x$ e $y$.
Calculando os pontos $P$ e $Q$, obtemos
\begin{align*}
  P & = \left( \frac{c+b}{2}, \frac{d}{2} \right), \\
  Q & = \left( \frac{x}{2}, \frac{y}{2} \right).
\end{align*}
A igualdade $P = Q$ implica $x = c + b$ e $y = d$.
Logo $(x,y) = (b + c, d)$, ou seja, $C = (b + c, d)$.
Portanto $C = B + \overrightarrow{AD}$ e $C = D + \overrightarrow{AB}$, ou seja, $\overrightarrow{BC} = \overrightarrow{AD}$ e $\overrightarrow{DC} = \overrightarrow{AB}$.
Isso implica que $ABCD$ é um paralelogramo.

\section*{Seção 15 -- Operações com Vetores}

\textbf{1.}
Dados os vetores $u$ e $v$, prove que as seguintes afirmações são equivalentes:
\begin{enumerate}[(a)]
  \item
    Uma combinação linear $\alpha u + \beta v$ só pode ser igual a zero quando $\alpha = 0$ e $\beta = 0$.
  \item
    Se $\alpha u + \beta v = \alpha' u + \beta' v$, então $\alpha = \alpha'$ e $\beta = \beta'$.
  \item
    Nenhum dos vetores $u$ e $v$ é múltiplo do outro.
  \item
    Para $u = (a, b)$ e $v = (a', b')$, temos $a b' - a' b \neq 0$.
  \item
    Todo vetor do plano é combinação linear de $u$ e $v$.
\end{enumerate}
(Neste exercício, devem ser provadas as implicações (a) $\Rightarrow$ (b) $\Rightarrow$ (c) $\Rightarrow$ (d) $\Rightarrow$ (e) $\Rightarrow$ (a).)

\vspace{\baselineskip}

\emph{Solução.}
(a) $\Rightarrow$ (b).
Se $\alpha u + \beta v = \alpha' u + \beta' v$, então $(\alpha - \alpha')u + (\beta - \beta') v = 0$.
Logo (a) implica que $\alpha - \alpha' = 0$ e $\beta - \beta' = 0$, ou seja, $\alpha = \alpha'$ e $\beta = \beta'$.

(b) $\Rightarrow$ (c).
Vamos provar que $-$(c) implica $-$(b).
Se existe $\lambda$ tal que $u = \lambda v$, então $\alpha u + \beta v = \alpha' u + \beta' v$ com $\alpha = 1$, $\beta = -\lambda$, $\alpha' = 0$ e $\beta' = 0$, onde $\alpha \neq \alpha'$ e $\beta \neq \beta'$, ou seja, a afirmação $-$(b) é verdadeira.

(c) $\Rightarrow$ (d).
Vamos provar que $-$(d) implica $-$(c).
Suponha que $\alpha \beta' - \alpha' \beta = 0$.
Se $\alpha = \alpha' = \beta = \beta'$, então $u = 0$ e $v = 0$, e portanto $u = \lambda v$ para todo $\lambda$, ou seja, a afirmação $-$(c) é verdadeira.
Se $\alpha = \beta = 0$, então $u = 0$, e portanto $u = 0v$.
Logo $-$(c) é verdadeira.
Se $\alpha \neq 0$ e $\beta \neq 0$, então $\beta'/\beta = \alpha'/\alpha = \lambda$ para alguma constante $\lambda$.
Logo $\alpha' = \lambda \alpha$ e $\beta' = \lambda \beta$, ou seja, $(\alpha', \beta') = \lambda (\alpha, \beta)$, ou seja $v = \lambda u$, ou seja, a afirmação $-$(c) é verdadeira.

(d) $\Rightarrow$ (e).
Sejam $u = (\alpha,\beta)$, $v = (\alpha',\beta')$ e $w = (\gamma,\delta)$.
Então a equação $w = xu + yv$ é equivalente ao sistema de equações
\begin{align*}
  \alpha x + \alpha' y & = \lambda \\
  \beta x + \beta' y & = \gamma.
\end{align*}
Como $\alpha \beta' - \alpha' \beta \neq 0$, esse sistema possui apenas uma solução.
De fato, a solução é
\[
  x = \frac{\beta' \gamma - \alpha' \delta}{\beta' \alpha - \alpha' \beta}, \qquad y = \frac{\beta \gamma - \alpha \delta}{\beta \alpha' - \alpha \beta'}.
\]

(e) $\Rightarrow$ (a).
Temos que $xu + yv = 0$ para $x$ e $y$ únicos.
Como $x = 0$ e $y = 0$ é solução desse sistema, essa deve ser a única solução, ou seja, a afirmação (a) é verdadeira.

\vspace{\baselineskip}

\textbf{7.}
Seja $P$ um ponto interior ao triângulo $ABC$ tal que $\overrightarrow{PA} \!+\! \overrightarrow{PB} + \overrightarrow{PC} = 0$.
Prove que as retas $AP$, $BP$ e $CP$ são medianas de $ABC$, logo $P$ é o ba\-ri\-cen\-tro desse triângulo.

\vspace{\baselineskip}

\emph{Solução.}
Seja $Q$ o ponto de intersecção da reta $BP$ com o segmento $AC$.
Observamos que $\overrightarrow{QA} = \alpha \overrightarrow{CA}$ para $\alpha \in \R$.
Logo
\[
  \overrightarrow{QC} = \overrightarrow{QA} + \overrightarrow{AC} = \alpha \overrightarrow{CA} - \overrightarrow{CA} = (\alpha - 1) \overrightarrow{CA}.
\]
Vamos provar que $Q$ é o ponto médio de $AC$, ou seja, vamos provar que $\alpha = 1/2$.
Escrevemos
\begin{align*}
  \overrightarrow{PA} & = \overrightarrow{PQ} + \overrightarrow{QA} = \overrightarrow{PQ} + \alpha \overrightarrow{CA}, \\
  \overrightarrow{PB} & = \overrightarrow{PQ} + \overrightarrow{QC} + \overrightarrow{CB} = \overrightarrow{PQ} + (\alpha - 1) \overrightarrow{CA} + \overrightarrow{CB}, \\
  \overrightarrow{PC} & = \overrightarrow{PQ} + \overrightarrow{QC} = \overrightarrow{PQ} + (\alpha - 1) \overrightarrow{CA}.
\end{align*}
Logo
\[
  \overrightarrow{PA} + \overrightarrow{PB} + \overrightarrow{PC} = 3 \overrightarrow{PQ} + (3\alpha - 2) \overrightarrow{CA} + \overrightarrow{CB}.
\]
Além disso,
\[
  \overrightarrow{BQ} = \overrightarrow{BC} + \overrightarrow{CQ} = - \overrightarrow{CB} - \overrightarrow{QC} = - \overrightarrow{CB} + (1 - \alpha) \overrightarrow{CA}.
\]
Para algum $\beta \in \R$, temos
\[
  \overrightarrow{PQ} = \beta \overrightarrow{BQ}.
\]
Portanto
\[
  \overrightarrow{PQ} = \beta \overrightarrow{BQ} = - \beta \overrightarrow{CB} + \beta (1 - \alpha) \overrightarrow{CA}.
\]
Consequentemente
\[
  \overrightarrow{PA} + \overrightarrow{PB} + \overrightarrow{PC} = (3 \beta (1 - \alpha) + 3\alpha - 2) \overrightarrow{CA} + (1 - 3 \beta) \overrightarrow{CB}.
\]
Por outro lado, temos $\overrightarrow{PA} + \overrightarrow{PB} + \overrightarrow{PC} = 0$.
Logo
\[
  (3 \beta (1 - \alpha) + 3\alpha - 2) \overrightarrow{CA} + (1 - 3 \beta) \overrightarrow{CB} = 0.
\]
Como $\overrightarrow{CA}$ e $\overrightarrow{CB}$ são linearmente independentes, essa igualdade implica
(veja o Exercício 1 da Seção 15)
\[
  (3 \beta(1 - \alpha) + 3\alpha - 2) = 0 \qquad \text{e} \qquad 1 - 3 \beta = 0.
\]
A segunda equação implica $\beta = 1/3$.
Substituindo esse valor de $\beta$ na primeira equação, obtemos $3(1/3)(1 - \alpha) + 3\alpha - 2 = 0$, ou seja, $\alpha = 1/2$.
Portanto $Q$ é o ponto médio de $AC$.
Renomeando os pontos, obtemos a demonstração para as medianas correspondentes aos outros vértices do triângulo.

\section*{Seção 16 -- Equação da Elipse}

\textbf{10.}
Quais são as tangentes à elipse $x^2 + 4y^2 = 32$ que têm inclinação igual a $1/2$?

\vspace{\baselineskip}

\emph{Solução.}
Uma reta com inclinação $1/2$ é dada por $y = (1/2)x + b$ para $b \in \R$.
Vamos determinar os valores de $b$ para os quais a reta $y = (1/2)x + b$ é tangente à elipse $x^2 + 4y^2 = 32$, ou seja, vamos determinar os valores de $b$ para os quais o sistema
\begin{align*}
  & x^2 + 4y^2 = 32 \\
  & y = (1/2)x + b
\end{align*}
tem apenas uma solução.
Substituindo a segunda equação na primeira e desenvolvendo, obtemos
\[
  2x^2 + 4bx + (4b^2 - 32) = 0.
\]
Essa equação possui apenas uma solução se, e somente se, o discriminante da equação é igual a zero, ou seja,
\[
  \Delta = -16b^2 + 16^2 = 0.
\]
Isso implica $b = \pm 4$.
Portanto, as retas tangentes são
\[
  y = \frac{1}{2} x - 4 \qquad \text{e} \qquad y = \frac{1}{2} x + 4.
\]

\section*{Seção 17 -- Equação da Hipérbole}

\textbf{2.}
Para todo ponto $P = (m,n)$ na hipérbole $H : x^2/a^2 - y^2/b^2 = 1$, mostre que a reta $r: (m/a^2)x - (n/b^2)y = 1$ tem apenas o ponto $P$ em comum com $H$.
A reta $r$ chama-se a \emph{tangente} a $H$ no ponto $P$.

\vspace{\baselineskip}

\emph{Solução.}
A reta $r$ é tangente à hipérbole $H$ no ponto $P$ se, e somente se, $x = m$ e $y = n$ é a única solução do sistema
\begin{align*}
  (m/a^2) x - (n/b^2) y & = 1 \\
  x^2/a^2 - y^2/b^2 & = 1.
\end{align*}
A primeira equação implica
\[
  x = \frac{a^2}{m} \left( 1 + \frac{n}{b^2} \right).
\]
Substituindo essa expressão para $x$ na segunda equação e desenvolvendo, obtemos
\[
  (a^2 n^2 - b^2 m^2) y^2 + b^2 a^2 2ny + b^4(a^2 - m^2) = 0.
\]
Como $P$ pertence à hipérbole, temos $a^2 n^2 - b^2 m^2 = -a^2 b^2$.
Substituindo essa expressão na equação anterior e simplificado, encontramos
\[
  -a^2 y^2 + a^2 2ny + b^2 (a^2 - m^2) = 0.
\]
Calculando o discriminante $\Delta$ dessa equação quadrática, obtemos
\[
  \Delta = 4 a^2 ( a^2 n^2 - b^2 m^2 + b^2 a^2 ) = 4 a^2 (-a^2 b^2 + b^2 a^2 ) = 4 a^2 (0) = 0.
\]
Nesse cálculo, usamos novamente que $P$ pertence a $H$.
Como $\Delta = 0$, a equação para $y$ possui apenas uma solução.
Associado a essa solução temos apenas um valor para $x$.
Portanto o sistema de equações possui apenas uma solução $(x,y)$, ou seja, a reta $r$ é tangente à hipérbole $H$.

\section*{Seção 20 -- Formas Quadráticas}

\textbf{1.}
Para cada uma das formas quadráticas abaixo, execute as seguintes tarefas:
\begin{enumerate}
  \item
    Escreva sua matriz e sua equação característica;
  \item
    Obtenha seus autovalores;
  \item
    Descreva suas linhas de nível;
  \item
    Ache autovetores unitários ortogonais $u$ e $u^*$;
  \item
    Determine os novos eixos em cujas coordenadas a forma quadrática se exprime como $A' s^2 + C' t^2$;
  \item
    Ache os focos da cônica $A' s^2 + C' t^2 = 1$ em termos das coordenadas $x$~e~$y$.
\end{enumerate}
As formas quadráticas são:
\begin{enumerate}[(a)]
  \item
    $\varphi(x,y) = x^2 + xy + y^2$.
  \item
    $\varphi(x,y) = xy$.
  \item
    $\varphi(x,y) = x^2 -6 xy +9 y^2$.
  \item
    $\varphi(x,y) = x^2 + xy - y^2$.
  \item
    $\varphi(x,y) = x^2 +2 xy -3 y^2$.
  \item
    $\varphi(x,y) = x^2 +24 xy -6 y^2$.
\end{enumerate}

\vspace{\baselineskip}

\emph{Solução.}
(a)
A matriz de $\varphi$ é
\[
  \begin{bmatrix}
    1 & 1/2 \\
    1/2 & 1
  \end{bmatrix}.
\]
A equação característica de $\varphi$ é $\lambda^2 - 2\lambda + 3/4 = 0$.
Logo os autovalores são $\lambda_1 = 3/2$ e $\lambda_2 = 1/2$.
Os autovetores unitários correspondentes são
\[
  u_1 = \left( \frac{1}{\sqrt{2}}, \frac{1}{\sqrt{2}} \right), \qquad u_2 = \left( \frac{1}{\sqrt{2}}, -\frac{1}{\sqrt{2}} \right).
\]
Tomamos $(\cos \theta, \sin \theta) = u_2$.
Consequentemente, se efetuarmos a mudança de variáveis
\begin{align*}
  x & = \frac{1}{\sqrt{2}} s + \frac{1}{\sqrt{2}} t, \\
  y & = -\frac{1}{\sqrt{2}} s + \frac{1}{\sqrt{2}} t,
\end{align*}
a forma quadrática assume a forma
\[
  \overline{\varphi}(s,t) = \frac{1}{2} s^2 + \frac{3}{2} t^2 = \frac{s^2}{2} + \frac{t^2}{2/3}.
\]
Para $d < 0$, as linhas de nível $\overline{\varphi}(s,t) = d$ são o conjunto vazio.
Para $d = 0$, a linha de nível $\overline{\varphi}(s,t) = d$ é o ponto $(0,0)$.
Para $d > 0$, as linhas de nível $\overline{\varphi}(s,t) = d$ são elipses.
Nesse caso, temos uma elipse com $c^2 = a^2 - b^2$, ou seja, $c = 2 \sqrt{d/3}$.
Portanto os focos da elipse são $(-c,0)$ e $(c,0)$, no sistema $s$ e $t$.
Em termos das coordenadas $x$ e $y$, os focos são, respectivamente,
\[
  \left( \frac{-\sqrt{2d}}{\sqrt{3}}, \frac{\sqrt{2d}}{\sqrt{3}} \right), \qquad \left( \frac{\sqrt{2d}}{\sqrt{3}}, -\frac{\sqrt{2d}}{\sqrt{3}} \right).
\]

(b)
A matriz de $\varphi$ é
\[
  \begin{bmatrix}
    0 & 1/2 \\
    1/2 & 0
  \end{bmatrix}.
\]
A equação característica de $\varphi$ é $\lambda^2 - 1/4 = 0$.
Logo os autovalores são $\lambda_1 = -1/2$ e $\lambda_2 = 1/2$.
Os autovetores unitários correspondentes são
\[
  u = \left( \frac{1}{\sqrt{2}}, \frac{-1}{\sqrt{2}} \right), \qquad u^* = \left( \frac{1}{\sqrt{2}}, \frac{1}{\sqrt{2}} \right).
\]
Consequentemente, se efetuarmos a mudança de variáveis
\begin{align*}
  x & = \frac{1}{\sqrt{2}} s + \frac{1}{\sqrt{2}} t, \\
  y & = \frac{1}{\sqrt{2}} s - \frac{1}{\sqrt{2}} t,
\end{align*}
a forma quadrática assume a forma
\[
  \overline{\varphi}(s,t) = -\frac{s^2}{2} + \frac{t^2}{2}.
\]
Para $d = 0$, a linha de nível $\overline{\varphi}(s,t) = d$ são as retas $t = \pm s$.
Para $d \neq 0$, as linhas de nível $\overline{\varphi}(s,t) = d$ são hipérboles.
Em particular, para $d < 0$, temos uma hipérbole com $c^2 = a^2 + b^2$, ou seja, $c = 2 \sqrt{|d|}$.
Portanto os focos da elipse são $(-c,0)$ e $(c,0)$, no sistema $s$ e $t$.
Em termos das coordenadas $x$ e $y$, os focos são, respectivamente,
\[
  ( -\sqrt{2|d|}, -\sqrt{2|d|}), \qquad ( \sqrt{2|d|}, \sqrt{2|d|} ).
\]

(c)
A matriz de $\varphi$ é
\[
  \begin{bmatrix}
    1 & -3 \\
    -3 & 9
  \end{bmatrix}.
\]
A equação característica de $\varphi$ é $\lambda^2 - 10\lambda = 0$.
Logo os autovalores são $\lambda_1 = 0$ e $\lambda_2 = 10$.
Os autovetores unitários correspondentes são
\[
  u = \left( \frac{3}{\sqrt{10}}, \frac{1}{\sqrt{10}} \right), \qquad u^* = \left( \frac{-1}{\sqrt{10}}, \frac{3}{\sqrt{10}} \right).
\]
Consequentemente, se efetuarmos a mudança de variáveis
\begin{align*}
  x & = \frac{3}{\sqrt{10}} s - \frac{1}{\sqrt{10}} t, \\
  y & = \frac{1}{\sqrt{10}} s + \frac{3}{\sqrt{10}} t,
\end{align*}
a forma quadrática assume a forma
\[
  \overline{\varphi}(s,t) = 10 t^2.
\]
Para $d < 0$, as linhas de nível $\overline{\varphi}(s,t) = d$ são o conjunto vazio.
Para $d = 0$, a linha de nível $\overline{\varphi}(s,t) = d$ é a reta horizontal $t = 0$ que passa pela origem.
Para $d > 0$, as linhas de nível $\overline{\varphi}(s,t) = d$ são o par de retas horizontais $t = \pm \sqrt{d/10}$.
Em termos das coordenadas $x$ e $y$, a reta $t = 0$ é dada por $x - 3y = 0$, e as retas $t = \pm \sqrt{d/10}$ são dadas por $x - 3y = \mp \sqrt{d}$.

(d)
A matriz de $\varphi$ é
\[
  \begin{bmatrix}
    1 & 1/2 \\
    1/2 & -1
  \end{bmatrix}.
\]
A equação característica de $\varphi$ é $\lambda^2 - 5/4 = 0$.
Logo os autovalores são $\lambda_1 = \sqrt{5}/2$ e $\lambda_2 = -\sqrt{5}/2$.
Os autovetores unitários correspondentes são
\[
  u = \left( \frac{1}{\sqrt{10 - 4\sqrt{5}}}, \frac{\sqrt{5}-2}{\sqrt{10 - 4\sqrt{5}}} \right), \qquad u^* = \left( \frac{2-\sqrt{5}}{\sqrt{10 - 4\sqrt{5}}}, \frac{1}{\sqrt{10 - 4\sqrt{5}}} \right).
\]
Consequentemente, se efetuarmos a mudança de variáveis
\begin{align*}
  x & = u_1 s - u_2 t, \\
  y & = u_2 s + u_1 t,
\end{align*}
onde $u_1$ e $u_2$ são as coordenadas de $u$, a forma quadrática assume a forma
\[
  \overline{\varphi}(s,t) = \frac{\sqrt{5}}{2} s^2 - \frac{\sqrt{5}}{2} t^2 = \frac{s^2}{2/\sqrt{5}} - \frac{t^2}{2/\sqrt{5}}.
\]
Para $d = 0$, as linhas de nível $\overline{\varphi}(s,t) = d$ são as retas $t = \pm s$.
Para $d \neq 0$, as linhas de nível $\overline{\varphi}(s,t) = d$ são hipérboles.
Em particular, para $d > 0$, temos uma hipérbole com $c^2 = a^2 + b^2$, ou seja, $c = 2\sqrt{d}/\sqrt{\sqrt{5}}$.
Portanto os focos da hipérbole são $(-c,0)$ e $(c,0)$, no sistema $s$ e $t$.
Em termos das coordenadas $x$ e $y$, os focos são, respectivamente,
\[
  \left( \frac{-2\sqrt{d}}{\sqrt{10 \sqrt{5} - 20}}, \frac{4\sqrt{d}-2\sqrt{5d}}{\sqrt{10 \sqrt{5} - 20}} \right), \qquad \left( \frac{2\sqrt{d}}{\sqrt{10 \sqrt{5} - 20}}, \frac{2\sqrt{5d}-4\sqrt{d}}{\sqrt{10 \sqrt{5} - 20}} \right).
\]

(e)
A matriz de $\varphi$ é
\[
  \begin{bmatrix}
    1 & 1 \\
    1 & -3
  \end{bmatrix}.
\]
A equação característica de $\varphi$ é $\lambda^2 + 2 \lambda - 4 = 0$.
Logo os autovalores são $\lambda_1 = -1-\sqrt{5}$ e $\lambda_2 = -1+\sqrt{5}$.
Os autovetores unitários correspondentes são
\[
  u = \left( \frac{1}{\sqrt{10 + 4\sqrt{5}}}, \frac{-\sqrt{5}-2}{\sqrt{10 + 4\sqrt{5}}} \right), \qquad u^* = \left( \frac{2+\sqrt{5}}{\sqrt{10 + 4\sqrt{5}}}, \frac{1}{\sqrt{10 + 4\sqrt{5}}} \right).
\]
Consequentemente, se efetuarmos a mudança de variáveis
\begin{align*}
  x & = u_1 s - u_2 t, \\
  y & = u_2 s + u_1 t,
\end{align*}
onde $u_1$ e $u_2$ são as coordenadas de $u$, a forma quadrática assume a forma
\[
  \overline{\varphi}(s,t) = \frac{s^2}{-1/(1+\sqrt{5})} + \frac{t^2}{1/(-1+\sqrt{5})}.
\]
Para $d = 0$, as linhas de nível $\overline{\varphi}(s,t) = d$ são as retas $t = \pm s (\sqrt{5} + 1)/2$.
Para $d \neq 0$, as linhas de nível $\overline{\varphi}(s,t) = d$ são hipérboles.
Em particular, para $d < 0$, temos uma hipérbole com $c^2 = a^2 + b^2$, ou seja, $c = \sqrt{3|d|}/2$.
Portanto os focos da hipérbole são $(-c,0)$ e $(c,0)$, no sistema $s$ e $t$.
Em termos das coordenadas $x$ e $y$, os focos são, respectivamente,
\[
  \left( \frac{-\sqrt{3|d|}}{2\sqrt{10 + 4\sqrt{5}}}, \frac{-\sqrt{15|d|}-2\sqrt{3|d|}}{2\sqrt{10 + 4\sqrt{5}}} \right), \ \ \left( \frac{2\sqrt{3|d|}+\sqrt{15|d|}}{2\sqrt{10 + 4\sqrt{5}}}, \frac{\sqrt{3|d|}}{2\sqrt{10 + 4\sqrt{5}}} \right)
\]

(f)
A matriz de $\varphi$ é
\[
  \begin{bmatrix}
    1 & 12 \\
    12 & -6
  \end{bmatrix}.
\]
A equação característica de $\varphi$ é $\lambda^2 + 5\lambda - 150 = 0$.
Logo os autovalores são $\lambda_1 = -15$ e $\lambda_2 = 10$.
Os autovetores unitários correspondentes são
\[
  u = \left( \frac{3}{5}, \frac{-4}{5} \right), \qquad u^* = \left( \frac{4}{5}, \frac{3}{5} \right).
\]
Consequentemente, se efetuarmos a mudança de variáveis
\begin{align*}
  x & = \frac{3}{5} s + \frac{4}{5} t, \\
  y & = -\frac{4}{5} s + \frac{3}{5} t,
\end{align*}
a forma quadrática assume a forma
\[
  \overline{\varphi}(s,t) = -15 s^2 + 10 t^2 = \frac{s^2}{-1/15} + \frac{t^2}{1/10}.
\]
Para $d = 0$, as linhas de nível $\overline{\varphi}(s,t) = d$ são as retas $t = \pm \sqrt{3/2} s$.
Para $d \neq 0$, as linhas de nível $\overline{\varphi}(s,t) = d$ são hipérboles.
Em particular, para $d < 0$, temos uma hipérbole com $c^2 = a^2 + b^2$, ou seja, $c = \sqrt{6|d|}/6$.
Nesse caso, os focos da hipérbole são $(-c,0)$ e $(c,0)$, no sistema $s$ e $t$.
Em termos das coordenadas $x$ e $y$, os focos são, respectivamente,
\[
  \left( -\frac{3\sqrt{6|d|}}{30}, \frac{4\sqrt{6|d|}}{30} \right), \qquad \left( \frac{3\sqrt{6|d|}}{30}, -\frac{4\sqrt{6
|d|}}{30} \right).
\]

\section*{Seção 23 -- Transformações Lineares}

\textbf{11.}
Uma transformação linear $T : \R^2 \to \R^2$ de posto $2$ transforma toda reta numa reta.
Prove isto.

\vspace{\baselineskip}

\emph{Solução.}
Seja $T(x,y) = (ax + by, cx + dy)$ e seja $M$ a matriz de $T$.
Como $M$ tem posto $2$, os vetores-coluna de $M$ são não-colineares.
Se $r$ é uma reta vertical, então $r$ é formada pelos pontos $(x,y) = (\alpha,t)$ para $t \in \R$.
Logo, os pontos
\[
  T(x,y) = T(\alpha, t) = (a \alpha + bt, c \alpha + dt) = \alpha(a, c) + t(b,d)
\]
para $t \in \R$ formam uma reta, pois $(b,d) \neq (0,0)$ (caso contrário teríamos $ad - bc = 0$, o que é impossível).
Se $r$ é uma reta não-vertical, então $r$ é o conjunto dos pontos $(x,y) = (t, \alpha t + \beta)$ para $t \in \R$.
Logo, os pontos
\[
  T(x,y) = T(t, \alpha t + \beta) = \beta(b,d) + t \big( (a,c) + \alpha (b,d) \big)
\]
para $t \in \R$ formam uma reta, pois não existe $\alpha$ tal que $(a,c) + \alpha(b,d) = 0$ (caso contrário $(a,c)$ e $(b,d)$ seriam colineares).

\vspace{\baselineskip}

\textbf{15.}
Seja $T : \R^2 \to \R^2$ uma transformação linear invertível.
Mostre que $T$ transforma retas paralelas em retas paralelas, portanto paralelogramos em paralelogramos.
E losangos?

\vspace{\baselineskip}

\emph{Solução.}
Seja $T(x,y) = (ax + by, cx + dy)$ e seja $M$ a matriz de $T$.
Como $T$ é invertível, para todo $(m,n) \in \R^2$ existe apenas um vetor $(x,y) \in \R^2$ tal que $T(x,y) = (m,n)$.
Dito de outra forma, o sistema de equações
\begin{align*}
  ax + by & = m \\
  cx + dy & = n
\end{align*}
possui apenas uma solução.
Portanto as retas $ax + by = m$ e $cx + dy = n$ são concorrentes.
Logo os vetores $(a,b)$ e $(c,d)$ são não-colineares, ou seja, $ad - bc \neq 0$.
Isso implica que os vetores $(a,c)$ e $(b,d)$ são não-colineares, ou seja, que a matriz de $M$ tem posto 2.
Pela solução do Exercício 11, para qualquer valor de $\alpha$, a transformação $T$ mapeia a reta $x = \alpha$ em uma reta paralela ao vetor $(b,d)$ que passa por $(a,c)$.
Isso mostra que $T$ transforma as retas paralelas $x = \alpha$ e $x = \alpha'$ em retas paralelas ao vetor $(b,d)$.
Pela solução do Exercício 11, a transformação $T$ mapeia a reta $y = \alpha x + \beta$ em uma reta paralela ao vetor $(a,c) + \alpha(b,d)$ que passa por $(a,c)$.
Analogamente, a transformação $T$ mapeia a reta $y = \alpha' x + \beta$ em uma reta paralela ao vetor $(a,c) + \alpha'(b,d)$ que passa por $(a,c)$.
Como $(a,b) + \alpha(c,d)$ e $(a,c) + \alpha'(b,d)$ são vetores colineares, concluímos que $T$ transforma retas não-verticais paralelas em retas não-verticais paralelas.
Além disso, concluímos que $T$ transforma paralelogramos em paralelogramos.
A transformação $T$ não mapeia losangos em losangos, em geral.
De fato, considere o quadrado cujos vértices são os pontos $A = (0,0)$, $B = (1,0)$, $C = (1,1)$ e $D = (0,1)$ (esse é um exemplo de losango).
Observamos que os os vetores unitários $\overrightarrow{AB} = (1,0)$ e $\overrightarrow{AD} = (0,1)$ são mapeados nos vetores $(a,c)$ e $(b,d)$, que não são unitários, em geral.
Logo o quadrado $ABCD$ não é transformado em um quadrado, em geral.

\vspace{\baselineskip}

\textbf{17.}
Dados $u = (1,2)$, $v = (3,4)$, $u' = (5,6)$ e $v' = (7,8)$, ache uma transformação linear $T : \mathbb{R}^2 \to \mathbb{R}^2$ tal que $Tu = u'$ e $Tv = v'$.

\vspace{\baselineskip}

\emph{Solução.}
Seja $T(x,y) = (ax + by, cx + dy)$, onde $a, b, c, d \in \R$.
Procuramos constantes $a$, $b$, $c$ e $d$ tais que $T(1,2) = (5,6)$ e $T(3,4) = (7,8)$, ou seja, $(a + 2b, c + 2d) = (5,6)$ e $(3a + 4b, 3c + 4d) = (7,8)$, ou seja, $a + 2b = 5$, $c + 2d = 6$ e $3a + 4b = 7$, $3c + 4d = 8$.
Obtemos portanto um sistema de quatro equações e quatro incógnitas, $a$, $b$, $c$ e $d$.
De fato, obtemos dois sistemas de duas equações e duas incógnitas, desacoplados:
\[
  \begin{aligned}
    a + 2b & = 5 \\
    3a + 4b & = 7
  \end{aligned}
  \qquad \qquad
  \begin{aligned}
    c + 2d & = 6 \\
    3c + 4d & = 8.
  \end{aligned}
\]
Resolvendo esses sistemas, obtemos $a = -3$, $b = 4$, $c = -4$ e $d = 5$.
Portanto, a transformação linear procurada é
\[
  T(x,y) = (-3x + 4y, -4x + 5y).
\]

\section*{Seção 24 -- Coordenadas no Espaço}

\textbf{5.}
Escreva a equação do plano vertical que passa pelos pontos $P\!=\!(2,3,4)$ e $Q = (1,1,758)$.

\vspace{\baselineskip}

\emph{Solução.}
O plano vertical que passa por $P$ e $Q$ deve conter todos os pontos da forma $(2,3,z)$ e $(1,1,z')$ para $z \in \R$ e $z' \in \R$.
Em particular, o plano vertical deve conter os pontos $P' = (2,3,0)$ e $Q' = (1,1,0)$.
Além disso, observamos que o plano vertical deve conter a reta $P'Q'$.
As coordenadas de $P'$ e $Q'$ no plano $\Pi_{xy}$ são $(2,3)$ e $(1,1)$.
Portanto $\overrightarrow{P'Q'} = (-1,-2)$ no plano $\Pi_{xy}$.
O vetor $v = (2,-1)$ é ortogonal a $\overrightarrow{P'Q'}$.
Logo a equação da reta $P'Q'$ no plano $\Pi_{xy}$ é $2x - y = c = 2(1) - 1(1) = 1$, ou seja, $2x - y = 1$.
O plano vertical que passa por $P$ e $Q$ é formado por todos os pontos $(x,y,z)$ tais que $2x - y = 1$.
Essa é a equação do plano.

\vspace{\baselineskip}

\textbf{7.}
Escreva a equação geral de um plano vertical.

\vspace{\baselineskip}

\emph{Solução.}
A equação geral de um plano vertical é $ax + by = c$, onde $a$, $b$ e $c$ são números reais.
De fato, o conjunto de todos os pontos $(x,y,z)$ tais que $ax + by = c$ forma um plano que contém o eixo $OZ$ ou é paralelo ao eixo $OZ$ (veja a solução do Exercício 5).

\section*{Seção 28 -- Vetores no Espaço}

\textbf{3.}
Seja $u = (a, b, c)$ um vetor unitário, com $abc \neq 0$.
Determine o valor de $t$ de modo que, pondo $v = (-bt, at, 0)$ e $w = (act, bct -1/t)$, os vetores $u$, $v$ e $w$ sejam unitários e mutuamente ortogonais.

\vspace{\baselineskip}

\emph{Solução.}
Como $u$ é unitário, temos $a^2 + b^2 + c^2 = 1$.
Observamos que $u \cdot v = 0$ e $v \cdot w = 0$ para qualquer valor de $t$.
Por outro lado, $u \cdot w = 0$ implica $t = \pm 1/\sqrt{a^2 + b^2}$.
Para esses valores de $t$, obtemos $\| v \|^2 = (b^2 + a^2)t^2 = 1$ e $\| w \|^2 = c^2 + a^2 + b^2 = 1$.
A condição $abc \neq 0$ pode ser substituída por $a^2 + b^2 \neq 0$.

\section*{Seção 29 -- Equação do Plano}

\textbf{2.}
Obtenha uma equação para o plano que contém $P$ e é perpendicular ao segmento de reta $AB$ nos seguintes casos:
\begin{enumerate}[(a)]
  \item
    $P = (0,0,0)$, $A = (1,2,3)$ e $B = (2,-1,2)$.
  \item
    $P = (1,1,-2)$, $A = (3,5,2)$ e $B = (7,1,12)$.
  \item
    $P = (3,3,3)$, $A = (2,2,2)$ e $B = (4,4,4)$.
  \item
    $P = (x_0, y_0, z_0)$, $A = (x_1,y_1,z_1)$ e $B = (x_2,y_2,z_2)$.
\end{enumerate}

\vspace{\baselineskip}

\emph{Solução.}
(a)
Observamos que o plano é perpendicular à reta $AB$ se e somente se o plano é perpendicular à reta $OB'$ com $B' = (1,-3,-1)$.
Logo uma equação para o plano é $x - 3y - z = d$ para alguma constante $d$.
Como $P$ pertence ao plano, devemos ter $1(0) - 3(0) -1(0) = d$, ou seja, $d=0$.
Portanto uma equação do plano é $x - 3y - z = 0$.

(b)
Procedendo como no item (a), obtemos a equação $4x - 4y + 10z = -20$.

(c)
Procedendo como no item (a), obtemos a equação $2x + 2y + 2z = 18$.

(d)
Procedendo como no item (a), obtemos a equação $(x_2 - x_1)x + (y_2 - y_1)y + (z_2 - z_1)z = (x_2 - x_1)x_0 + (y_2 - y_1)y_0 + (z_2 - z_1)z_0$.

\vspace{\baselineskip}

\textbf{4.}
Sejam $A = (-1,1,2)$, $B = (2,3,5)$ e $C = (1,3,-2)$.
Obtenha uma equação para o plano que contém a reta $AB$ e o ponto $C$.

\vspace{\baselineskip}

\emph{Solução.}
Procuramos um vetor $v = (a,b,c)$ tal que $\langle v, \overrightarrow{AB} \rangle = 0$ e $\langle v, \overrightarrow{AC} \rangle = 0$.
Calculamos $\overrightarrow{AB} = (3,2,3)$ e $\overrightarrow{AC} = (2,2,-4)$.
Com isso obtemos o seguinte sistema de equações para $(a,b,c)$:
\begin{align*}
  3a + 2b + 3c & = 0 \\
  2a + 2b - 4c & = 0.
\end{align*}
Escrevemos
\begin{align*}
  3a + 2b & = -3c \\
  2a + 2b & = 4c
\end{align*}
e resolvemos para $a$ e $b$ considerando $c$ como um parâmetro.
Obtemos que $(-7c, 9c, c)$ para $c \in \R$ são as soluções do sistema original.
Em particular, o vetor $v = (-7, 9, 1)$ é solução do sistema.
Portanto, uma equação do plano é $-7x + 9y + z = d$ para alguma constante $d$.
Como $A$ pertence ao plano, devemos ter $-7(-1) + 9(1) + 1(2) = d$, ou seja, $d = 18$.
Portanto, uma equação para o plano é $-7x + 9y + z = 18$.

\section*{Seção 31 - Sistemas de Equações Lineares com Três Incógnitas}

\textbf{1.}
Para cada um dos sistemas a seguir, decida se existem ou não soluções.
No caso afirmativo, exiba todas as soluções do sistema em termos de um ou dois parâmetros independentes.
\[
  \text{(a)} \
  \begin{aligned}
    x + 2y + 3z & = 4 \\
    2x + 3y + 4z & = 5
  \end{aligned}
  \qquad
  \text{(b)} \
  \begin{aligned}
    2x - y + 5z & = 3 \\
    4x - 2y + 10z & = 5
  \end{aligned}
  \qquad
  \text{(c)} \
  \begin{aligned}
    6x - 4y + 12z & = 2 \\
    9x - 6y + 18z & = 3
  \end{aligned}
\]

\vspace{\baselineskip}

\emph{Solução.}
(a)
Observamos que os vetores $l_1 = (1,2,3)$ e $l_2 = (2,3,4)$ não são colineares.
Logo os planos definidos pelas equações se intersectam segundo uma reta, ou seja, o sistema possui soluções.
A matriz aumentada do sistema é
\[
  \begin{bmatrix}
    1 & 2 & 3 & 4 \\
    2 & 3 & 4 & 5
  \end{bmatrix}.
\]
Escalonando essa matriz, obtemos
\[
  \begin{bmatrix}
    1 & 0 & -1 & -2 \\
    0 & 1 & 2 & 3
  \end{bmatrix}.
\]
Portanto as soluções do sistema são $x = -2 + t$, $y = 3 - 2t$, $z = t$ para $t \in \R$.

(b)
Observamos que os vetores $l_1 = (2,-1,5)$ e $l_2 = (4,-2,10)$ são colineares e os vetores $L_1 = (2,-1,5,3)$ e $L_2 = (4,-2,10,5)$ não são colineares.
Logo os planos definidos pelas equações são paralelos, ou seja, o sistema não possui soluções.

(c)
Observamos que os vetores $l_1 = (6,-4,12)$ e $l_2 = (9,-6,18)$ são colineares e os vetores $L_1 = (6,-4,12,2)$ e $L_2 = (9,-6,18,3)$ são colineares.
Logo os planos definidos pelas equações são coincidentes, ou seja, o sistema possui soluções.
A matriz aumentada do sistema é
\[
  \begin{bmatrix}
    6 & -4 & 12 & 2 \\
    9 & -6 & 18 & 3
  \end{bmatrix}.
\]
Escalonando essa matriz obtemos
\[
  \begin{bmatrix}
    1 & -2/3 & 2 & 1/3 \\
    0 & 0 & 0 & 0
  \end{bmatrix}.
\]
Portanto as soluções do sistema são $x = 1/3 + (2/3)s - 2t$, $y = s$, $z = t$ para $s, t \in \R$.

\section*{Seção 41 - Mudança de Coordenadas no Espaço}

\textbf{1.}
Ache números $\alpha$, $\beta$ de modo que os múltiplos $\alpha m$ e $\beta n$ das matrizes abaixo sejam matrizes ortogonais
\[
  m =
  \begin{bmatrix}
    2 & -2 & 1 \\
    1 & 2 & 2 \\
    2 & 1 & -2
  \end{bmatrix},
  \qquad n =
  \begin{bmatrix}
    6 & 3 & 2 \\
    -3 & 2 & 6 \\
    2 & -6 & 3
  \end{bmatrix}.
\]

\vspace{\baselineskip}

\emph{Solução.}
Procuramos $\alpha$ tal que $(\alpha m)(\alpha m)^T = I$, ou seja, $\alpha^2 (m m^T) = I$.
Calculando $mm^T$, obtemos
\[
  \alpha^2 (mm^T) = \alpha^2
  \begin{bmatrix}
    9 & 0 & 0 \\
    0 & 9 & 0 \\
    0 & 0 & 9
  \end{bmatrix}.
\]
Portanto, a condição para $\beta$ é $\beta^2 9 = 1$, ou seja, $\beta = \pm 1/3$.

Procuramos $\beta$ tal que $(\beta n)(\beta n)^T = I$, ou seja, $\beta^2 (n n^T) = I$.
Calculando $nn^T$, obtemos
\[
  \beta^2 (nn^T) = \beta^2
  \begin{bmatrix}
    49 & 0 & 0 \\
    0 & 49 & 0 \\
    0 & 0 & 49
  \end{bmatrix}.
\]
Portanto, a condição para $\beta$ é $\beta^2 49 = 1$, ou seja, $\beta = \pm 1/7$.

\end{document}
