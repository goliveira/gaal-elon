\documentclass[a4paper,11pt]{article}
\usepackage[brazilian]{babel}
\usepackage[utf8]{inputenc}
\usepackage[T1]{fontenc}
\usepackage{amsmath}
\usepackage{amssymb}
\usepackage{amsthm}
\usepackage{indentfirst}
\usepackage{enumerate}

\newcommand{\R}{\mathbb{R}}

\title{Exercícios Resolvidos do Livro\\Geometria Analítica e Álgebra Linear\\de Elon Lages Lima\\(Segunda Edição--Oitava Impressão)}
\author{Gustavo de Oliveira}

\begin{document}

\maketitle

\section*{Seção 2 -- Coordenadas no Plano}

\textbf{S2.E8.}
Com argumento análogo ao do exercício anterior, determine o simétrico do ponto $A = (x,y)$ em relação à diagonal $\Delta' = \{(x,-x) \ | \ x \in \R\}$.

\vspace{\baselineskip}

\emph{Solução.}
Seja $A''$ o simétrico de $A$ em relação à diagonal $\Delta'$.
Procedendo como na solução do exercício anterior obtemos $A'' = (-y,-x)$.

\section*{Seção 5 -- Escolhendo o Sistema de Coordenadas}

\textbf{S5.E2.}
O triângulo $ABC$ é equilátero e cada lado mede $l$.
Num sistema de coordenadas em que a origem é equidistante de $A$, $B$ e $C$, e o ponto $C$ está sobre o eixo $OY$, quais são as coordenadas dos três vértices?

\vspace{\baselineskip}

\emph{Solução.}
Tomando o sistema de coordenadas $OXY$ da forma sugerida, obtemos os triângulos $ABO$, $BCO$ e $CAO$ com $O = (0,0)$, $A = (-a,-b)$, $B = (a,-b)$ e $C = (0,c)$.
Esses três triângulos são isóceles e congruentes.
Por exemplo, temos $ABO$ com $\hat{A} = \hat{B} = 30^0$ e $\hat{O} = 120^o$.
Logo
\[
  d(A,O) \cos 30^o = l/2,
\]
ou seja, $d(A,O) \sqrt{3}/2 = l/2$, ou ainda, $d(A,O) = l/\sqrt{3}$.
Seja $M$ o ponto médio de $AB$.
Então
\[
  d(M,O) = d(A,O) \sin 30^o = (l/\sqrt{3})(1/2) = l/(2\sqrt{3}).
\]
Logo
\begin{align*}
  a & = l/2, \\
  b & = d(M,O) = l/(2\sqrt{3}), \\
  c & = d(O,C) = d(O,A) = l/\sqrt{3}.
\end{align*}
Portanto
\[
  A = \left(-\frac{l}{2}, -\frac{l}{2\sqrt{3}} \right), \qquad B = \left( \frac{l}{2}, \frac{l}{2\sqrt{3}} \right), \qquad C = \left(0, \frac{l}{\sqrt{3}} \right).
\]

\section*{Seção 7 -- As Equações da Reta}

\textbf{S7.E10.}
Sejam $A = (1,2)$, $B = (2,4)$ e $C = (3,-1)$.
Ache as equações da mediana e da altura do triângulo $ABC$ que partem do vértice $A$.

\vspace{\baselineskip}

\emph{Solução.}
A mediana que parte do vértice $A$ é o segmento $AM$ onde $M$ é o ponto médio do lado $BC$.
Calculando $M$, obtemos $M = (5/2,3/2)$.
A reta que passa por $A$ e $M$ tem inclinação $[(3/2)-2]/[(5/2)-1] = -1/3$, ou seja, a equação dessa reta é da forma $y = -(1/3)x + b$.
Calculando $b$, obtemos $b = 7/3$.
Logo a equação da mediana $AM$ é $y = -(1/3)x + 7/3$.

A reta que contém a altura do vértice $A$ é a reta perpendicular ao lado $BC$ passando por $A$.
O lado $BC$ é paralelo a $OC'$ com $C' = (1,-5)$.
Logo a equação da altura tem a forma $x - 5y = b$.
Como a reta passa por $A$, devemos ter $1-5(2) = b$, ou seja, $b = -9$.
Portanto a equação da altura é $x - 5y = -9$.

\vspace{\baselineskip}

\textbf{S7.E22.}
Qual é a distância entre as paralelas $x - 3y = 4$ e $2x - 6y = 1$?

\vspace{\baselineskip}

\emph{Solução.}
Sejam $r$ e $s$ as retas definidas por $x - 3y = 4$ e $2x - 6y = 1$, respectivamente.
Seja $t$ a reta perpendicular a $r$ (e portanto a $s$) que passa pela origem.
Uma equação para $t$ é $3x + y = 0$.
Calculamos $\{P\} = r \cap t$, ou seja, resolvemos o sistema $x-3y = 4$, $3x + y = 0$.
A solução desse sistema é $x = 2/5$ e $y = -6/5$.
Portanto $P = (2/5, -6/5)$.
Calculamos $\{Q\} = s \cap t$, ou seja, resolvemos o sistema $2x - 6y = 1$, $3x + y = 0$.
A solução desse sistema é $x = 1/20$ e $y = -3/20$.
Portanto $Q = (1/20, -3/20)$.
Observamos que $d(r,s) = d(P,Q)$.
Calculando $d(P,Q)$, concluímos que
$d(r,s) = 7/(2 \sqrt{10})$.

\vspace{\baselineskip}

\textbf{S7.E33.}
Ache uma representação paramétrica para a reta $5x - 2y = 1$.

\vspace{\baselineskip}

\emph{Solução.}
Tomamos $x = t$ e procuramos $y$ tal que $5t - 2y = 1$.
Obtemos $y = -1/2 + (5/2)t$.
Portanto $t \mapsto (t, -1/2 + (5/2)t)$ é uma parametrização para a reta.

\section*{Seção 9 -- Distância de um Ponto a uma Reta}

\textbf{S9.E2.}
Qual é o raio da circunferência que tem centro em $P = (4,1)$ e é tangente à reta $3x + 7y = 2$?

\vspace{\baselineskip}

\emph{Solução.}
Seja $Q$ o ponto em que a reta toca a circunferência.
Observamos que o segmento $PQ$ é perpendicular à reta e o comprimento de $PQ$ é igual a distância de $P$ à reta, que por sua vez é igual ao raio $r$ da circunferência.
Portanto
\[
  r = d(P,\mathrm{reta}) = \frac{|2-(3(4) + 7(1))|}{\sqrt{3^2 + 7^2}} = \frac{17}{\sqrt{58}}.
\]

\vspace{\baselineskip}

\section*{Seção 10 -- Área de um Triângulo}

\textbf{S10.E5.}
Calcule a área do triângulo cujos vértices são intersecções de duas das retas $x+y=0$, $x-y=0$ e $2x+y=3$.

\vspace{\baselineskip}

\emph{Solução.}
Calculamos os vértices do triângulo resolvendo sistemas de equações lineares:
A intersecção das retas $x+y=0$ e $x-y=0$ é o ponto $O = (0,0)$.
A intersecção das retas $x+y=0$ e $2x+y=3$ é o ponto $A = (3,-3)$.
A intersecção das retas $x-y=0$ e $2x+y=3$ é o ponto $B = (1,1)$.
Portanto
\[
  \text{Área}_{OAB} =
  \frac{1}{2} \left|
  \begin{vmatrix}
    1-0 & 3-0 \\
    1-0 & -3-0
  \end{vmatrix}
  \right|
  =
  \frac{1}{2} \left|
  \begin{vmatrix}
    1 & 3 \\
    1 & -3
  \end{vmatrix}
  \right|
  = \frac{1}{2} |-3-3| = 3.
\]

\vspace{\baselineskip}

\section*{Seção 12 -- Equação da Circunferência}

\textbf{S12.E4.}
Qual é a equação da circunferência que passa pelos pontos $A = (1,2)$, $B = (3,4)$ e tem o centro sobre o eixo $OY$?

\vspace{\baselineskip}

\emph{Solução.}
Como o centro da circunferência está sobre o eixo $OY$, a equação da circunferência tem a forma $x^2 + (y-b)^2 = r^2$.
Como $A$ e $B$ pertencem à circunferência, devemos ter
\begin{align*}
  1 + (2 - b)^2 & = r^2 \\
  9 + (4 - b)^2 & = r^2.
\end{align*}
Resolvendo esse sistema, obtemos $b = 5$ e $r = \sqrt{10}$.
Portanto a equação da circunferência é $x^2 + (y - 5)^2 = 10$.

\vspace{\baselineskip}

\textbf{S12.E5.}
Escreva a equação da circunferência que tem centro no ponto $P = (2,5)$ e é tangente à reta $y = 3x + 1$.

\vspace{\baselineskip}

\emph{Solução.}
A reta tem equação $3x - y = -1$.
Logo
\[
  d(P,\mathrm{reta}) = \frac{|-1-(3(2) + (-1)5)|}{\sqrt{3^2 + 1^2}} = \frac{2}{\sqrt{10}}.
\]
Portanto a equação da circunferência é $(x-2)^2 + (y-5)^2 = 2/5$.

\vspace{\baselineskip}

\textbf{S12.E9.}
A tangente, no ponto $P$, à circunferência de centro $O$ e raio $3$ é paralela à reta $y = -2x + 1$.
Quais são as coordenadas de $P$?
E se o raio da circunferência fosse $5$?

\vspace{\baselineskip}

\emph{Solução.}
Primeiro, procuramos $b$ tal que $r: 2x + y = b$ passe por $P$ (note que $r$ é paralela à reta $y = -2x + 1$).
Calculamos
\[
  d(O,r) = \frac{|b - (2(0) + 1(0))|}{\sqrt{2^2 + 1^2}} = \frac{|b|}{\sqrt{5}}.
\]
Suponha que $d(O,r) = \rho$ (onde $\rho$ é conhecida).
Então $|b|/\sqrt{5} = \rho$.
Logo $b = \pm \rho \sqrt{5}$.

Agora, observamos que $P$ é o ponto de interseção da reta $r$ com a reta $-x + 2y = 0$ (que é a reta perpendicular a $r$ passando por $O$).
Calculando o ponto $P$, obtemos
\[
  P = \Big( \frac{2b}{5}, \frac{b}{5} \Big) = \Big( \pm 2\rho \frac{\sqrt{5}}{5}, \pm \rho \frac{\sqrt{5}}{5} \Big).
\]
Portanto, se $\rho = 3$ temos
\[
  P = \Big( \pm \frac{5 \sqrt{5}}{5}, \pm \frac{3\sqrt{5}}{5} \Big),
\]
e se $\rho = 5$ obtemos
\[
  P = ( \pm 2\sqrt{5}, \pm \sqrt{5}).
\]

\section*{Seção 13 -- Reconhecimento da Equação da\\Circunferência}

\textbf{S13.E4.}
Completando os quadrados, decida se cada uma das equações abaixo define uma circunferência, um ponto ou o conjunto vazio:
\begin{enumerate}[(a)]
  \item
    $2x^2 + 2y^2 - 3x + y - 1 = 0$.
  \item
    $-x^2 - y^2 + 6x - 4y + 3 = 0$.
  \item
    $x^2 + y^2 - 10x + 2y + 26 = 0$.
  \item
    $4x^2 + 4y^2 - 4x - 8y + 21 = 0$.
\end{enumerate}

\vspace{\baselineskip}

\emph{Solução.}
(a)
Completando os quadrados, obtemos
\[
  \left( x - \frac{3}{4} \right)^2 + \left( y + \frac{1}{4} \right)^2 = \left( \frac{3}{2\sqrt{2}} \right)^2.
\]
Portanto a equação define uma circunferência.

(b)
Completando os quadrados, obtemos
\[
  (x-3)^2 + (y+2)^2 = 4^2.
\]
Portanto a equação define uma circunferência.

(c)
Completando os quadrados, obtemos
\[
  (x-5)^2 + (y-1)^2 = 0.
\]
Portanto a equação define um ponto.

(d)
Completando os quadrados, obtemos
\[
  \left( x - \frac{1}{2} \right)^2 + (y - 1)^2 = -4.
\]
Portanto a equação define o conjunto vazio.

\section*{Seção 14 -- Vetores no Plano}

\textbf{S14.E2.}
Prove geometricamente que um quadrilátero é um paralelogramo se, e somente se, suas diagonais se cortam mutuamente ao meio.

\vspace{\baselineskip}

\emph{Solução.}
($\Rightarrow$)
Suponha que o quadrilátero (com vértices consecutivos) $ABCD$ é um paralelogramo.
Então $\overrightarrow{AB} + \overrightarrow{BC} + \overrightarrow{CD} + \overrightarrow{DA} = 0$.
Além disso, existem constantes $\alpha$ e $\beta$ tais que $\overrightarrow{AB} = \alpha \overrightarrow{DC}$ e $\overrightarrow{AD} = \beta \overrightarrow{BC}$.
Logo $\overrightarrow{AB} = -\alpha \overrightarrow{CD}$ e $\overrightarrow{DA} = -\beta \overrightarrow{BC}$.
Portanto $-\alpha \overrightarrow{CD} + \overrightarrow{BC} + \overrightarrow{CD} -\beta \overrightarrow{BC} = 0$, ou seja, $(1-\alpha) \overrightarrow{CD} + (1-\beta) \overrightarrow{BC} = 0$.
Como $ABCD$ é um quadrilátero, os vetores $\overrightarrow{CD}$ e $\overrightarrow{BC}$ não são colineares.
Logo a última igualdade implica $1 - \alpha = 0$ e $1 - \beta = 0$, isto é, $\alpha = 1$ e $\beta = 1$.
Portanto $\overrightarrow{AB} = \overrightarrow{DC}$ e $\overrightarrow{AD} = \overrightarrow{BC}$.
Agora, temos duas relações que envolvem as diagonais do paralelogramo:
\begin{align*}
  \overrightarrow{AB} + \overrightarrow{BC} + \overrightarrow{CA} = 0, \\
  \overrightarrow{AB} + \overrightarrow{BD} + \overrightarrow{DA} = 0.
\end{align*}
Vamos eliminar dessas equações os vetores correspondentes aos lados do quadrilátero escrevendo-os em termos das direções diagonais.
Usando as relações acima, obtemos
\begin{align*}
  \overrightarrow{AB} + \overrightarrow{BC} + \overrightarrow{CA} = 0, \\
  \overrightarrow{AB} + \overrightarrow{BD} - \overrightarrow{BC} = 0.
\end{align*}
Somando as duas igualdades, chegamos a
\[
  2\overrightarrow{AB} + \overrightarrow{CA} + \overrightarrow{BD} = 0.
\]
Como $A$, $B$, $C$ e $D$ não são colineares, as retas $AC$ e $BD$ que contém as diagonais são concorrentes em um ponto $M$.
Sejam $\lambda$ e $\gamma$ constantes tais que $\overrightarrow{AM} = \lambda \overrightarrow{AC}$ e $\overrightarrow{MB} = \gamma \overrightarrow{DB}$.
Calculamos
\begin{align*}
  2\overrightarrow{AB} + \overrightarrow{CA} + \overrightarrow{BD} & = 0 \\
  2(\overrightarrow{AM} + \overrightarrow{MB}) + \overrightarrow{CA} + \overrightarrow{BD} & = 0 \\
  2(\lambda \overrightarrow{AC} + \gamma \overrightarrow{DB}) - \overrightarrow{AC} - \overrightarrow{DB} & = 0 \\
  (2\lambda - 1) \overrightarrow{AC} + (2\gamma - 1) \overrightarrow{DB} & = 0.
\end{align*}
Essa iqualdade implica $2\lambda - 1 = 0$ e $2\gamma - 1 = 0$, ou seja, $\lambda = 1/2$ e $\gamma = 1/2$, como queríamos provar.

($\Leftarrow$)
Suponha que as diagonais de $ABCD$ se bissectam em um ponto $M$.
Então $\overrightarrow{AM} = \overrightarrow{MC}$ e $\overrightarrow{BM} = \overrightarrow{MD}$.
Mas $\overrightarrow{BC} = \overrightarrow{BM} + \overrightarrow{MC}$ e $\overrightarrow{AD} = \overrightarrow{AM} + \overrightarrow{MD}$.
Logo $\overrightarrow{BC} = \overrightarrow{AD}$.
Analogamente provamos que $\overrightarrow{AB} = \overrightarrow{DC}$.
Portanto $ABCD$ é um paralelogramo.

\section*{Seção 15 -- Operações com Vetores}

\textbf{S15.E1.}
Dados os vetores $u$ e $v$, prove que as seguintes afirmações são equivalentes:
\begin{enumerate}[(a)]
  \item
    Uma combinação linear $\alpha u + \beta v$ só pode ser igual a zero quando $\alpha = 0$ e $\beta = 0$.
  \item
    Se $\alpha u + \beta v = \alpha' u + \beta' v$, então $\alpha = \alpha'$ e $\beta = \beta'$.
  \item
    Nenhum dos vetores $u$ e $v$ é múltiplo do outro.
  \item
    Para $u = (a, b)$ e $v = (a', b')$, temos $a b' - a' b \neq 0$.
  \item
    Todo vetor do plano é combinação linear de $u$ e $v$.
\end{enumerate}
(Neste exercício, devem ser provadas as implicações (a) $\Rightarrow$ (b) $\Rightarrow$ (c) $\Rightarrow$ (d) $\Rightarrow$ (e) $\Rightarrow$ (a).)

\vspace{\baselineskip}

\emph{Solução.}
(a) $\Rightarrow$ (b).
Suponha (a), ou seja, suponha que $\alpha u + \beta v = 0$ implica $\alpha = 0$ e $\beta = 0$.
Se $\alpha u + \beta v = \alpha' u + \beta' v$, então $(\alpha - \alpha')u + (\beta - \beta') v = 0$.
Logo (a) implica $\alpha - \alpha' = 0$ e $\beta - \beta' = 0$, ou seja, $\alpha = \alpha'$ e $\beta = \beta'$.

(b) $\Rightarrow$ (c).
Vamos provar que $-$(c) implica $-$(b).
Se existe $\lambda$ tal que $u = \lambda v$, então $\alpha u + \beta v = \alpha' u + \beta' v$ com $\alpha = 1$, $\beta = -\lambda$, $\alpha' = 0$ e $\beta' = 0$, onde $\alpha \neq \alpha'$ e $\beta \neq \beta'$, ou seja, a afirmação $-$(b) é verdadeira.

(c) $\Rightarrow$ (d).
Vamos provar que $-$(d) implica $-$(c).
Suponha $a b' - a' b = 0$.
Se $a = a' = b = b'$, então $u = 0$ e $v = 0$, e portanto $u = \lambda v$ para todo $\lambda$, ou seja, a afirmação $-$(c) é verdadeira.
Se $a = b = 0$, então $u = 0$, e portanto $u = 0v$.
Logo $-$(c) é verdadeira.
Se $a \neq 0$ e $b \neq 0$, então $b'/b = a'/a = \lambda$ em que $\lambda$ é uma constante.
Logo $a' = \lambda a$ e $b' = \lambda b$, ou seja, $(a', b') = \lambda (a, b)$, ou seja $v = \lambda u$, ou seja, a afirmação $-$(c) é verdadeira.

(d) $\Rightarrow$ (e).
Sejam $u = (a,b)$, $v = (a',b')$ e $w = (\gamma,\delta)$.
Então a equação $w = xu + yv$ é equivalente ao sistema de equações
\begin{align*}
  a x + a' y & = \lambda \\
  b x + b' y & = \gamma.
\end{align*}
Como $a b' - a' b \neq 0$, esse sistema possui apenas uma solução.
De fato, a solução é
\[
  x = \frac{b' \gamma - a' \delta}{b' a - a' b}, \qquad y = \frac{b \gamma - a \delta}{b a' - a b'}.
\]

(e) $\Rightarrow$ (a).
Suponha (e).
Então, em particular, o vetor $0$ é combinação de $u$ e $v$.
Logo $xu + yv = 0$ para $x$ e $y$ únicos.
Como $x = 0$ e $y = 0$ é solução desse sistema, essa deve ser a única solução, ou seja, a afirmação (a) é verdadeira.

\vspace{\baselineskip}

\textbf{S15.E7.}
Seja $P$ um ponto interior ao triângulo $ABC$ tal que $\overrightarrow{PA} \!+\! \overrightarrow{PB} + \overrightarrow{PC} = 0$.
Prove que as retas $AP$, $BP$ e $CP$ são medianas de $ABC$, logo $P$ é o ba\-ri\-cen\-tro desse triângulo.

\vspace{\baselineskip}

\emph{Solução.}
Seja $Q$ o ponto de intersecção da reta $BP$ com o segmento $AC$.
Observamos que $\overrightarrow{QA} = \alpha \overrightarrow{CA}$ para $\alpha \in \R$.
Logo
\[
  \overrightarrow{QC} = \overrightarrow{QA} + \overrightarrow{AC} = \alpha \overrightarrow{CA} - \overrightarrow{CA} = (\alpha - 1) \overrightarrow{CA}.
\]
Vamos provar que $Q$ é o ponto médio de $AC$, ou seja, vamos provar que $\alpha = 1/2$.
Escrevemos
\begin{align*}
  \overrightarrow{PA} & = \overrightarrow{PQ} + \overrightarrow{QA} = \overrightarrow{PQ} + \alpha \overrightarrow{CA}, \\
  \overrightarrow{PB} & = \overrightarrow{PQ} + \overrightarrow{QC} + \overrightarrow{CB} = \overrightarrow{PQ} + (\alpha - 1) \overrightarrow{CA} + \overrightarrow{CB}, \\
  \overrightarrow{PC} & = \overrightarrow{PQ} + \overrightarrow{QC} = \overrightarrow{PQ} + (\alpha - 1) \overrightarrow{CA}.
\end{align*}
Logo
\[
  \overrightarrow{PA} + \overrightarrow{PB} + \overrightarrow{PC} = 3 \overrightarrow{PQ} + (3\alpha - 2) \overrightarrow{CA} + \overrightarrow{CB}.
\]
Além disso,
\[
  \overrightarrow{BQ} = \overrightarrow{BC} + \overrightarrow{CQ} = - \overrightarrow{CB} - \overrightarrow{QC} = - \overrightarrow{CB} + (1 - \alpha) \overrightarrow{CA}.
\]
Agora, para algum $\beta \in \R$, temos
\[
  \overrightarrow{PQ} = \beta \overrightarrow{BQ}.
\]
Assim
\[
  \overrightarrow{PQ} = \beta \overrightarrow{BQ} = - \beta \overrightarrow{CB} + \beta (1 - \alpha) \overrightarrow{CA}.
\]
Portanto
\[
  \overrightarrow{PA} + \overrightarrow{PB} + \overrightarrow{PC} = (3 \beta (1 - \alpha) + 3\alpha - 2) \overrightarrow{CA} + (1 - 3 \beta) \overrightarrow{CB}.
\]
Por outro lado, temos $\overrightarrow{PA} + \overrightarrow{PB} + \overrightarrow{PC} = 0$.
Logo
\[
  (3 \beta (1 - \alpha) + 3\alpha - 2) \overrightarrow{CA} + (1 - 3 \beta) \overrightarrow{CB} = 0.
\]
Como $\overrightarrow{CA}$ e $\overrightarrow{CB}$ são linearmente independentes, essa igualdade implica
(veja o Exercício 1 da Seção 15)
\[
  (3 \beta(1 - \alpha) + 3\alpha - 2) = 0 \qquad \text{e} \qquad 1 - 3 \beta = 0.
\]
A segunda equação implica $\beta = 1/3$.
Substituindo esse valor de $\beta$ na primeira equação, obtemos $3(1/3)(1 - \alpha) + 3\alpha - 2 = 0$, ou seja, $\alpha = 1/2$.
Portanto $Q$ é o ponto médio de $AC$.
As demonstrações para as medianas correspondentes aos outros vértices do triângulo são idênticas.
Basta renomear os pontos.

\vspace{\baselineskip}

\textbf{S15.E9.}
Mostre que se os vetores $u$ e $v$ têm o mesmo comprimento então $u + v$ e $u - v$ são ortogonais.
E a recíproca?

\vspace{\baselineskip}

\emph{Solução.}
Observamos que
\[
  \langle u + v, u - v \rangle = |u|^2 + \langle v, u \rangle - \langle u, v \rangle - |v|^2 = |u|^2 - |v|^2 = (|u| + |v|)(|u| - |v|).
\]

Se $|u| = |v|$, então $\langle u + v, u - v \rangle = 0$.
Logo $u + v$ e $u - v$ são ortogonais.

Por outro lado, se $u + v$ e $u - v$ são ortogonais, então $\langle u + v, u - v \rangle = 0$, logo $(|u|+|v|)(|u|-|v|) = 0$.
Essa igualdade implica $|u| = -|v|$, o que é impossível (exceto se $u = 0$ e $v = 0$), ou $|u| = |v|$.
Portanto $|u| = |v|$.

\section*{Seção 16 -- Equação da Elipse}

\textbf{S16.E10.}
Quais são as tangentes à elipse $x^2 + 4y^2 = 32$ que têm inclinação igual a $1/2$?

\vspace{\baselineskip}

\emph{Solução.}
Uma reta com inclinação $1/2$ é dada por $y = (1/2)x + b$ para $b \in \R$.
Vamos determinar os valores de $b$ para os quais essa reta é tangente à elipse $x^2 + 4y^2 = 32$, ou seja, vamos determinar os valores de $b$ para os quais o sistema
\begin{align*}
  & x^2 + 4y^2 = 32 \\
  & y = (1/2)x + b
\end{align*}
possui apenas uma solução.
Substituindo a segunda equação da primeira e desenvolvendo, obtemos
\[
  2x^2 + 4bx + (4b^2 - 32) = 0.
\]
Essa equação possui apenas uma solução se e somente se o discriminante da equação é igual a zero, ou seja,
\[
  \Delta = -16b^2 + 16^2 = 0.
\]
Isso implica $b = \pm 4$.
Portanto, as retas tangentes à elipse são
\[
  y = \frac{1}{2} x - 4 \qquad \text{e} \qquad y = \frac{1}{2} x + 4.
\]

\section*{Seção 17 -- Equação da Hipérbole}

\textbf{S17.E2.}
Para todo ponto $P = (m,n)$ na hipérbole $H : x^2/a^2 - y^2/b^2 = 1$, mostre que a reta $r: (m/a^2)x - (n/b^2)y = 1$ tem apenas o ponto $P$ em comum com $H$.
A reta $r$ chama-se a \emph{tangente} a $H$ no ponto $P$.

\vspace{\baselineskip}

\emph{Solução.}
A reta $r$ é tangente à hipérbole $H$ no ponto $P$ se e somente se $x = m$ e $y = n$ é a única solução do sistema
\begin{align*}
  (m/a^2) x - (n/b^2) y & = 1 \\
  x^2/a^2 - y^2/b^2 & = 1.
\end{align*}
A primeira equação implica
\[
  x = \frac{a^2}{m} \left( 1 + \frac{n}{b^2} \right).
\]
Substituindo essa expressão para $x$ na segunda equação e desenvolvendo, obtemos
\[
  (a^2 n^2 - b^2 m^2) y^2 + b^2 a^2 2ny + b^4(a^2 - m^2) = 0.
\]
Como $P$ pertence à hipérbole, temos $a^2 n^2 - b^2 m^2 = -a^2 b^2$.
Substituindo essa expressão na equação anterior e simplificado, encontramos
\[
  -a^2 y^2 + a^2 2ny + b^2 (a^2 - m^2) = 0.
\]
Calculando o discriminante $\Delta$ dessa equação quadrática, obtemos
\[
  \Delta = 4 a^2 ( a^2 n^2 - b^2 m^2 + b^2 a^2 ) = 4 a^2 (-a^2 b^2 + b^2 a^2 ) = 4 a^2 (0) = 0.
\]
Nesse cálculo, usamos novamente que $P$ pertence a $H$.
Como $\Delta = 0$, a equação para $y$ possui apenas uma solução.
Associado a essa solução temos apenas um valor para $x$.
Portanto o sistema de equações possui apenas uma solução $(x,y)$, ou seja, a reta $r$ é tangente à hipérbole $H$.

\section*{Seção 23 -- Transformações Lineares}

\textbf{S23.E11.}
Uma transformação linear $T : \R^2 \to \R^2$ de posto $2$ transforma toda reta numa reta.
Prove isto.

\vspace{\baselineskip}

\emph{Solução.}
Seja $T(x,y) = (ax + by, cx + dy)$ e seja $M$ a matriz de $T$.
Como $M$ tem posto $2$, os vetores-coluna de $M$ são não-colineares.
Se $r$ é uma reta vertical, então $r$ é formada pelos pontos $(x,y) = (\alpha,t)$ para $t \in \R$.
Logo, os pontos
\[
  T(x,y) = T(\alpha, t) = (a \alpha + bt, c \alpha + dt) = \alpha(a, c) + t(b,d)
\]
para $t \in \R$ formam uma reta, pois $(b,d) \neq (0,0)$ (caso contrário teríamos $ad - bc = 0$, o que é impossível).
Se $r$ é uma reta não-vertical, então $r$ é o conjunto dos pontos $(x,y) = (t, \alpha t + \beta)$ para $t \in \R$.
Logo, os pontos
\[
  T(x,y) = T(t, \alpha t + \beta) = \beta(b,d) + t \big( (a,c) + \alpha (b,d) \big)
\]
para $t \in \R$ formam uma reta, pois não existe $\alpha$ tal que $(a,c) + \alpha(b,d) = 0$ (caso contrário $(a,c)$ e $(b,d)$ seriam colineares).

\vspace{\baselineskip}

\textbf{S23.E15.}
Seja $T : \R^2 \to \R^2$ uma transformação linear invertível.
Mostre que $T$ transforma retas paralelas em retas paralelas, portanto paralelogramos em paralelogramos.
E losangos?

\vspace{\baselineskip}

\emph{Solução.}
Seja $T(x,y) = (ax + by, cx + dy)$ e seja $M$ a matriz de $T$.
Como $T$ é invertível, para todo $(m,n) \in \R^2$ existe apenas um vetor $(x,y) \in \R^2$ tal que $T(x,y) = (m,n)$.
Dito de outra forma, o sistema de equações
\begin{align*}
  ax + by & = m \\
  cx + dy & = n
\end{align*}
possui apenas uma solução.
Portanto as retas $ax + by = m$ e $cx + dy = n$ são concorrentes.
Logo os vetores $(a,b)$ e $(c,d)$ são não-colineares, ou seja, $ad - bc \neq 0$.
Isso implica que os vetores $(a,c)$ e $(b,d)$ são não-colineares, ou seja, que a matriz de $M$ tem posto 2.
Pela solução do Exercício S23.E11, para qualquer valor de $\alpha$, a transformação $T$ mapeia a reta $x = \alpha$ em uma reta paralela ao vetor $(b,d)$ que passa por $(a,c)$.
Isso mostra que $T$ transforma as retas paralelas $x = \alpha$ e $x = \alpha'$ em retas paralelas ao vetor $(b,d)$.
Pela solução do Exercício S23.E11, a transformação $T$ mapeia a reta $y = \alpha x + \beta$ em uma reta paralela ao vetor $(a,c) + \alpha(b,d)$ que passa por $(a,c)$.
Analogamente, a transformação $T$ mapeia a reta $y = \alpha' x + \beta$ em uma reta paralela ao vetor $(a,c) + \alpha'(b,d)$ que passa por $(a,c)$.
Como $(a,b) + \alpha(c,d)$ e $(a,c) + \alpha'(b,d)$ são vetores colineares, concluímos que $T$ transforma retas não-verticais paralelas em retas não-verticais paralelas.
Além disso, concluímos que $T$ transforma paralelogramos em paralelogramos.
A transformação $T$ não mapeia losangos em losangos, em geral.
De fato, considere o quadrado cujos vértices são os pontos $A = (0,0)$, $B = (1,0)$, $C = (1,1)$ e $D = (0,1)$ (esse é um exemplo de losango).
Observamos que os os vetores unitários $\overrightarrow{AB} = (1,0)$ e $\overrightarrow{AD} = (0,1)$ são mapeados nos vetores $(a,c)$ e $(b,d)$, que não são unitários, em geral.
Logo o quadrado $ABCD$ não é transformado em um quadrado, em geral.

\vspace{\baselineskip}

\textbf{S23.E17.}
Dados $u = (1,2)$, $v = (3,4)$, $u' = (5,6)$ e $v' = (7,8)$, ache uma transformação linear $T : \mathbb{R}^2 \to \mathbb{R}^2$ tal que $Tu = u'$ e $Tv = v'$.

\vspace{\baselineskip}

\emph{Solução.}
Seja $T(x,y) = (ax + by, cx + dy)$, onde $a, b, c, d \in \R$.
Procuramos constantes $a$, $b$, $c$ e $d$ tais que $T(1,2) = (5,6)$ e $T(3,4) = (7,8)$, ou seja, $(a + 2b, c + 2d) = (5,6)$ e $(3a + 4b, 3c + 4d) = (7,8)$, ou seja, $a + 2b = 5$, $c + 2d = 6$ e $3a + 4b = 7$, $3c + 4d = 8$.
Obtemos portanto um sistema de quatro equações e quatro incógnitas, $a$, $b$, $c$ e $d$.
De fato, obtemos dois sistemas de duas equações e duas incógnitas, desacoplados:
\[
  \begin{aligned}
    a + 2b & = 5 \\
    3a + 4b & = 7
  \end{aligned}
  \qquad \qquad
  \begin{aligned}
    c + 2d & = 6 \\
    3c + 4d & = 8.
  \end{aligned}
\]
Resolvendo esses sistemas, encontramos $a = -3$, $b = 4$, $c = -4$ e $d = 5$.
Portanto, a transformação linear procurada é
\[
  T(x,y) = (-3x + 4y, -4x + 5y).
\]

\section*{Seção 24 -- Coordenadas no Espaço}

\textbf{S24.E5.}
Escreva a equação do plano vertical que passa pelos pontos $P\!=\!(2,3,4)$ e $Q = (1,1,758)$.

\vspace{\baselineskip}

\emph{Solução.}
O plano vertical que passa por $P$ e $Q$ deve conter todos os pontos da forma $(2,3,z)$ e $(1,1,z')$ para $z \in \R$ e $z' \in \R$.
Em particular, o plano vertical deve conter os pontos $P' = (2,3,0)$ e $Q' = (1,1,0)$.
Além disso, observamos que o plano vertical deve conter a reta $P'Q'$.
As coordenadas de $P'$ e $Q'$ no plano $\Pi_{xy}$ são $(2,3)$ e $(1,1)$.
Portanto $\overrightarrow{P'Q'} = (-1,-2)$ no plano $\Pi_{xy}$.
O vetor $v = (2,-1)$ é ortogonal a $\overrightarrow{P'Q'}$.
Logo a equação da reta $P'Q'$ no plano $\Pi_{xy}$ é $2x - y = c = 2(1) - 1(1) = 1$, ou seja, $2x - y = 1$.
Portanto, o plano vertical que passa por $P$ e $Q$ é formado por todos os pontos $(x,y,z)$ tais que $2x - y = 1$.

\vspace{\baselineskip}

\textbf{S24.E7.}
Escreva a equação geral de um plano vertical.

\vspace{\baselineskip}

\emph{Solução.}
A equação geral de um plano vertical é $ax + by = c$, onde $a$, $b$ e $c$ são números reais.
De fato, o conjunto de todos os pontos $(x,y,z)$ tais que $ax + by = c$ forma um plano que contém o eixo $OZ$ ou é paralelo ao eixo $OZ$ (veja a solução do Exercício S24.E5).

\section*{Seção 25 -- As Equações Paramétricas de uma Reta}

\textbf{S25.E6.}
Dados $A = (1,2,3)$ e $B = (4,5,6)$, determine os pontos em que a reta $AB$ corta os planos $\Pi_{xy}$, $\Pi_{yz}$, $\Pi_{zx}$.

\vspace{\baselineskip}

\emph{Solução.}
As equações paramétricas da reta $AB$ são $x = 1 + 3t$, $y = 2 + 3t$, $z = 3 + 3t$.
Na interseção da reta com o plano $\Pi_{xy}$, temos $z = 0$, ou seja, $3 + 3t = 0$.
Isso implica em $t = -1$.
O ponto correspondente é $P = (-2,-1,0)$.
Na interseção da reta com o plano $\Pi_{yz}$, temos $x = 0$, ou seja, $1 + 3t = 0$.
Isso implica em $t = -1/3$.
O ponto correspondente é $Q = (0,1,2)$.
Na interseção da reta com o plano $\Pi_{zx}$, temos $y = 0$, ou seja, $2 + 3t = 0$.
Isso implica em $t = -2/3$.
O ponto correspondente é $R = (-1,0,1)$.

\section*{Seção 28 -- Vetores no Espaço}

\textbf{S28.E3.}
Seja $u = (a, b, c)$ um vetor unitário, com $abc \neq 0$.
Determine o valor de $t$ de modo que, pondo $v = (-bt, at, 0)$ e $w = (act, bct -1/t)$, os vetores $u$, $v$ e $w$ sejam unitários e mutuamente ortogonais.

\vspace{\baselineskip}

\emph{Solução.}
Como $u$ é unitário, temos $a^2 + b^2 + c^2 = 1$.
Observamos que $u \cdot v = 0$ e $v \cdot w = 0$ para qualquer valor de $t$.
Por outro lado, $u \cdot w = 0$ implica $t = \pm 1/\sqrt{a^2 + b^2}$.
Para esses valores de $t$, obtemos $\| v \|^2 = (b^2 + a^2)t^2 = 1$ e $\| w \|^2 = c^2 + a^2 + b^2 = 1$.
A condição $abc \neq 0$ pode ser substituída por $a^2 + b^2 \neq 0$.

\section*{Seção 29 -- Equação do Plano}

\textbf{S29.E2.}
Obtenha uma equação para o plano que contém o ponto $P$ e é perpendicular ao segmento de reta $AB$ nos seguintes casos:
\begin{enumerate}[(a)]
  \item
    $P = (0,0,0)$, $A = (1,2,3)$ e $B = (2,-1,2)$.
  \item
    $P = (1,1,-2)$, $A = (3,5,2)$ e $B = (7,1,12)$.
  \item
    $P = (3,3,3)$, $A = (2,2,2)$ e $B = (4,4,4)$.
  \item
    $P = (x_0, y_0, z_0)$, $A = (x_1,y_1,z_1)$ e $B = (x_2,y_2,z_2)$.
\end{enumerate}

\vspace{\baselineskip}

\emph{Solução.}
(a)
Observamos que o plano é perpendicular à reta $AB$ se e somente se o plano é perpendicular à reta $OB'$ com $B' = (1,-3,-1)$.
Logo uma equação para o plano é $x - 3y - z = d$ para alguma constante $d$.
Como $P$ pertence ao plano, devemos ter $1(0) - 3(0) -1(0) = d$, ou seja, $d=0$.
Portanto uma equação do plano é $x - 3y - z = 0$.

(b)
Procedendo como no item (a), obtemos a equação $4x - 4y + 10z = -20$.

(c)
Procedendo como no item (a), obtemos a equação $2x + 2y + 2z = 18$.

(d)
Procedendo como no item (a), obtemos a equação
\[
  (x_2 - x_1)x + (y_2 - y_1)y + (z_2 - z_1)z = (x_2 - x_1)x_0 + (y_2 - y_1)y_0 + (z_2 - z_1)z_0.
\]

\vspace{\baselineskip}

\textbf{S29.E4.}
Sejam $A = (-1,1,2)$, $B = (2,3,5)$ e $C = (1,3,-2)$.
Obtenha uma equação para o plano que contém a reta $AB$ e o ponto $C$.

\vspace{\baselineskip}

\emph{Solução.}
Procuramos um vetor $v = (a,b,c)$ tal que $\langle v, \overrightarrow{AB} \rangle = 0$ e $\langle v, \overrightarrow{AC} \rangle = 0$.
Calculamos $\overrightarrow{AB} = (3,2,3)$ e $\overrightarrow{AC} = (2,2,-4)$.
Com isso obtemos o seguinte sistema de equações para $(a,b,c)$:
\begin{align*}
  3a + 2b + 3c & = 0 \\
  2a + 2b - 4c & = 0.
\end{align*}
Escrevemos
\begin{align*}
  3a + 2b & = -3c \\
  2a + 2b & = 4c
\end{align*}
e resolvemos para $a$ e $b$ considerando $c$ como um parâmetro.
Obtemos que $(-7c, 9c, c)$ para $c \in \R$ são as soluções do sistema original.
Em particular, o vetor $v = (-7, 9, 1)$ é solução do sistema.
Portanto, uma equação do plano é $-7x + 9y + z = d$ para alguma constante $d$.
Como $A$ pertence ao plano, devemos ter $-7(-1) + 9(1) + 1(2) = d$, ou seja, $d = 18$.
Portanto, uma equação para o plano é $-7x + 9y + z = 18$.

\section*{Seção 31 - Sistemas de Equações Lineares com Três Incógnitas}

\textbf{S31.E1.}
Para cada um dos sistemas a seguir, decida se existem ou não soluções.
No caso afirmativo, exiba todas as soluções do sistema em termos de um ou dois parâmetros independentes.
\[
  \text{(a)} \
  \begin{aligned}
    x + 2y + 3z & = 4 \\
    2x + 3y + 4z & = 5
  \end{aligned}
  \qquad
  \text{(b)} \
  \begin{aligned}
    2x - y + 5z & = 3 \\
    4x - 2y + 10z & = 5
  \end{aligned}
  \qquad
  \text{(c)} \
  \begin{aligned}
    6x - 4y + 12z & = 2 \\
    9x - 6y + 18z & = 3
  \end{aligned}
\]

\vspace{\baselineskip}

\emph{Solução.}
(a)
Observamos que os vetores $l_1 = (1,2,3)$ e $l_2 = (2,3,4)$ não são colineares.
Logo os planos definidos pelas equações se intersectam segundo uma reta, ou seja, o sistema possui soluções.
A matriz aumentada do sistema é
\[
  \begin{bmatrix}
    1 & 2 & 3 & 4 \\
    2 & 3 & 4 & 5
  \end{bmatrix}.
\]
Escalonando essa matriz, obtemos
\[
  \begin{bmatrix}
    1 & 0 & -1 & -2 \\
    0 & 1 & 2 & 3
  \end{bmatrix}.
\]
Portanto as soluções do sistema são $x = -2 + t$, $y = 3 - 2t$, $z = t$ para $t \in \R$.

(b)
Vemos que os vetores $l_1 = (2,-1,5)$ e $l_2 = (4,-2,10)$ são colineares e os vetores $L_1 = (2,-1,5,3)$ e $L_2 = (4,-2,10,5)$ não são colineares.
Logo os planos definidos pelas equações são paralelos, ou seja, o sistema não possui soluções.

(c)
Observamos que os vetores $l_1 = (6,-4,12)$ e $l_2 = (9,-6,18)$ são colineares e os vetores $L_1 = (6,-4,12,2)$ e $L_2 = (9,-6,18,3)$ são colineares.
Logo os planos definidos pelas equações são coincidentes, ou seja, o sistema possui soluções.
A matriz aumentada do sistema é
\[
  \begin{bmatrix}
    6 & -4 & 12 & 2 \\
    9 & -6 & 18 & 3
  \end{bmatrix}.
\]
Escalonando essa matriz, obtemos
\[
  \begin{bmatrix}
    1 & -2/3 & 2 & 1/3 \\
    0 & 0 & 0 & 0
  \end{bmatrix}.
\]
Portanto as soluções são $x = 1/3 + (2/3)s - 2t$, $y = s$, $z = t$ para $s, t \in \R$.

\section*{Seção 41 - Mudança de Coordenadas no Espaço}

\textbf{S41.E1.}
Ache números $\alpha$, $\beta$ de modo que os múltiplos $\alpha m$ e $\beta n$ das matrizes abaixo sejam matrizes ortogonais
\[
  m =
  \begin{bmatrix}
    2 & -2 & 1 \\
    1 & 2 & 2 \\
    2 & 1 & -2
  \end{bmatrix},
  \qquad n =
  \begin{bmatrix}
    6 & 3 & 2 \\
    -3 & 2 & 6 \\
    2 & -6 & 3
  \end{bmatrix}.
\]

\vspace{\baselineskip}

\emph{Solução.}
Procuramos $\alpha$ tal que $(\alpha m)(\alpha m)^T = I$, ou seja, $\alpha^2 (m m^T) = I$.
Calculando $mm^T$, obtemos
\[
  \alpha^2 (mm^T) = \alpha^2
  \begin{bmatrix}
    9 & 0 & 0 \\
    0 & 9 & 0 \\
    0 & 0 & 9
  \end{bmatrix}.
\]
Portanto devemos ter $\alpha^2 9 = 1$, ou seja, $\beta = \pm 1/3$.

Procuramos $\beta$ tal que $(\beta n)(\beta n)^T = I$, ou seja, $\beta^2 (n n^T) = I$.
Calculando $nn^T$, obtemos
\[
  \beta^2 (nn^T) = \beta^2
  \begin{bmatrix}
    49 & 0 & 0 \\
    0 & 49 & 0 \\
    0 & 0 & 49
  \end{bmatrix}.
\]
Portanto devemos ter $\beta^2 49 = 1$, ou seja, $\beta = \pm 1/7$.

\end{document}
