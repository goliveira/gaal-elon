\documentclass[a4paper,11pt]{article}
\usepackage[brazilian]{babel}
\usepackage[T1]{fontenc}
% \documentclass[a4paper,11pt]{article}
% \usepackage[brazilian]{babel}
% \usepackage[T1]{fontenc}
\usepackage[utf8]{inputenc}
\usepackage{subfiles}
\usepackage{amsmath,amssymb}
\usepackage{amsthm}
\usepackage{commath}
\usepackage{framed}
\usepackage{enumerate}
\usepackage{hyperref}

\usepackage{fancyhdr}
\pagestyle{fancyplain}
\fancyhf{}
\renewcommand{\headrulewidth}{0pt}
\fancyfoot[EOL]{\footnotesize Gustavo de Oliveira}
\fancyfoot[EOC]{\thepage}
\fancyfoot[EOR]{\footnotesize \today}

\newtheorem{teorema}{Teorema}
\newtheorem{theorem}{Theorem}
\theoremstyle{definition}
\newtheorem{exercicio}{Exercício}
\newtheorem{lema}{Lema}
\newtheorem{lemma}{Lemma}

\newcommand{\re}{\mathrm{Re}\,}
\newcommand{\im}{\mathrm{Im}\,}

\newcommand{\N}{\mathbb{N}}
\newcommand{\Z}{\mathbb{Z}}
\newcommand{\R}{\mathbb{R}}
\newcommand{\C}{\mathbb{C}}
\newcommand{\F}{\mathbb{F}}
\newcommand{\E}{\mathcal{E}}

\newcommand{\concept}[1]{\textbf{#1}}
\newcommand{\setl}[1]{\{#1\}}
\newcommand{\setp}[2]{\{#1\mid#2\}}

\setlength{\parskip}{0.5em}


\title{GAAL -- Seção 2 -- Exercício 2}
\author{\empty}
\date{\empty}
\newcommand\onlyinsubfileone\maketitle

\begin{document}

\onlyinsubfileone

\begin{exercicio-gaal}[E2.S2]
  O conjunto $r$ formado pelos pontos $(x,5)$ cujas ordenadas são iguais a $5$ é uma reta paralela ao eixo $OX$.
  Determine o simétrico do ponto $P = (3,-2)$ em relação à reta $r$.
\end{exercicio-gaal}

\begin{proof}[Solução]
  A reta $s$ perpendicular a $r$ passando por $P$ é formada pelos pontos $(3,y)$ cujas abcissas são iguais a $3$.
  O ponto $P$ está à distância $5-(-2) = 7$ da reta $r$.
  A intersecção da reta $r$ com a reta $s$ é o ponto $(3,5)$.
  Logo o simétrico de $P$ em relação à reta $r$ é o ponto $P' = (3,5+7) = (3,12)$.
\end{proof}

\end{document}
