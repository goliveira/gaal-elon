\documentclass[a4paper,11pt]{article}
\usepackage[brazilian]{babel}
\usepackage[T1]{fontenc}
% \documentclass[a4paper,11pt]{article}
% \usepackage[brazilian]{babel}
% \usepackage[T1]{fontenc}
\usepackage[utf8]{inputenc}
\usepackage{subfiles}
\usepackage{amsmath,amssymb}
\usepackage{amsthm}
\usepackage{commath}
\usepackage{framed}
\usepackage{enumerate}
\usepackage{hyperref}

\usepackage{fancyhdr}
\pagestyle{fancyplain}
\fancyhf{}
\renewcommand{\headrulewidth}{0pt}
\fancyfoot[EOL]{\footnotesize Gustavo de Oliveira}
\fancyfoot[EOC]{\thepage}
\fancyfoot[EOR]{\footnotesize \today}

\newtheorem{teorema}{Teorema}
\newtheorem{theorem}{Theorem}
\theoremstyle{definition}
\newtheorem{exercicio}{Exercício}
\newtheorem{lema}{Lema}
\newtheorem{lemma}{Lemma}

\newcommand{\re}{\mathrm{Re}\,}
\newcommand{\im}{\mathrm{Im}\,}

\newcommand{\N}{\mathbb{N}}
\newcommand{\Z}{\mathbb{Z}}
\newcommand{\R}{\mathbb{R}}
\newcommand{\C}{\mathbb{C}}
\newcommand{\F}{\mathbb{F}}
\newcommand{\E}{\mathcal{E}}

\newcommand{\concept}[1]{\textbf{#1}}
\newcommand{\setl}[1]{\{#1\}}
\newcommand{\setp}[2]{\{#1\mid#2\}}

\setlength{\parskip}{0.5em}


\title{GAAL -- Seção 2 -- Exercício 1}
\author{\empty}
\date{\empty}
\newcommand\onlyinsubfileone\maketitle

\begin{document}

\onlyinsubfileone

\begin{exercicio-gaal}[E1.S2]
  Diz-se que o ponto $A'$ é o simétrico do ponto $A$ em relação à reta $r$ quando $r$ é a mediatriz do segmento $AA'$.
  Sabendo que $A = (x,y)$, determine os simétricos de $A$ em relação aos eixos $OX$ e $OY$.
\end{exercicio-gaal}

\begin{proof}[Solução]
  Sejam $A'$ e $A''$ os simétricos de $A$ em relação aos eixos $OX$ e $OY$, respectivamente.
  Fazendo uma figura, é simples verificar que o eixo $OX$ é a mediatriz do segmento $AA'$ com $A' = (x,-y)$ e o eixo $OY$ é a mediatriz do segmento $AA''$ com $A''=(-x,y)$.
\end{proof}

\end{document}
