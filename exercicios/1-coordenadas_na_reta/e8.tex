\documentclass[a4paper,11pt]{article}
\usepackage[brazilian]{babel}
\usepackage[T1]{fontenc}
% \documentclass[a4paper,11pt]{article}
% \usepackage[brazilian]{babel}
% \usepackage[T1]{fontenc}
\usepackage[utf8]{inputenc}
\usepackage{subfiles}
\usepackage{amsmath,amssymb}
\usepackage{amsthm}
\usepackage{commath}
\usepackage{framed}
\usepackage{enumerate}
\usepackage{hyperref}

\usepackage{fancyhdr}
\pagestyle{fancyplain}
\fancyhf{}
\renewcommand{\headrulewidth}{0pt}
\fancyfoot[EOL]{\footnotesize Gustavo de Oliveira}
\fancyfoot[EOC]{\thepage}
\fancyfoot[EOR]{\footnotesize \today}

\newtheorem{teorema}{Teorema}
\newtheorem{theorem}{Theorem}
\theoremstyle{definition}
\newtheorem{exercicio}{Exercício}
\newtheorem{lema}{Lema}
\newtheorem{lemma}{Lemma}

\newcommand{\re}{\mathrm{Re}\,}
\newcommand{\im}{\mathrm{Im}\,}

\newcommand{\N}{\mathbb{N}}
\newcommand{\Z}{\mathbb{Z}}
\newcommand{\R}{\mathbb{R}}
\newcommand{\C}{\mathbb{C}}
\newcommand{\F}{\mathbb{F}}
\newcommand{\E}{\mathcal{E}}

\newcommand{\concept}[1]{\textbf{#1}}
\newcommand{\setl}[1]{\{#1\}}
\newcommand{\setp}[2]{\{#1\mid#2\}}

\setlength{\parskip}{0.5em}


\title{GAAL -- Seção 1 -- Exercício 8}
\author{\empty}
\date{\empty}
\newcommand\onlyinsubfileone\maketitle

\begin{document}

\onlyinsubfileone

\begin{exercicio-gaal}[E8.S1]
  Sejam $A$, $B$, $C$, $D$ pontos dispostos nesta ordem sobre um eixo~$E$.
  Esboce os gráficos das funções $\varphi, f, g : E \to \R$ dadas por
  \begin{align*}
    \varphi(X) & = d(X,A) + d(X,B), \\
    f(X) & = d(X,A) + d(X,B) + d(X,C), \\
    g(X) & = d(X,A) + d(X,B) + d(X,C) + d(X,D).
  \end{align*}
\end{exercicio-gaal}

\begin{proof}[Solução]
  Por exemplo, tomamos $A$, $B$, $C$ e $D$ com coordenadas $0$, $1$, $3$ e $7$, respectivamente.
  Seja $x$ a coordenada de $X$.
  Então
  \begin{align*}
    \psi(x) & = |x| + |x-1|, \\
    f(x) & = |x| + |x-1| + |x-3|, \\
    g(x) & = |x| + |x-1| + |x-3| + |x-7|.
  \end{align*}
  Para ver o gráfico dessas funções, visite \texttt{https://sagecell.sagemath.org} e execute o código
\begin{verbatim}
a = 0
b = 1
c = 3
d = 7
m = -5
n = 8
phi(x) = abs(x-a) + abs(x-b)
f(x) = phi(x) + abs(x-c)
g(x) = f(x) + abs(x-d)
p1 = plot(phi(x), (x,m,n), color="blue")
p2 = plot(f(x), (x,m,n), color="red")
p3 = plot(g(x), (x,m,n), color="green")
p = p1 + p2 + p3
show(p, axes_labels=["x", "y"])
\end{verbatim}
\end{proof}

\end{document}
