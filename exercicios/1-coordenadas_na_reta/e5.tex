\documentclass[a4paper,11pt]{article}
\usepackage[brazilian]{babel}
\usepackage[T1]{fontenc}
% \documentclass[a4paper,11pt]{article}
% \usepackage[brazilian]{babel}
% \usepackage[T1]{fontenc}
\usepackage[utf8]{inputenc}
\usepackage{subfiles}
\usepackage{amsmath,amssymb}
\usepackage{amsthm}
\usepackage{commath}
\usepackage{framed}
\usepackage{enumerate}
\usepackage{hyperref}

\usepackage{fancyhdr}
\pagestyle{fancyplain}
\fancyhf{}
\renewcommand{\headrulewidth}{0pt}
\fancyfoot[EOL]{\footnotesize Gustavo de Oliveira}
\fancyfoot[EOC]{\thepage}
\fancyfoot[EOR]{\footnotesize \today}

\newtheorem{teorema}{Teorema}
\newtheorem{theorem}{Theorem}
\theoremstyle{definition}
\newtheorem{exercicio}{Exercício}
\newtheorem{lema}{Lema}
\newtheorem{lemma}{Lemma}

\newcommand{\re}{\mathrm{Re}\,}
\newcommand{\im}{\mathrm{Im}\,}

\newcommand{\N}{\mathbb{N}}
\newcommand{\Z}{\mathbb{Z}}
\newcommand{\R}{\mathbb{R}}
\newcommand{\C}{\mathbb{C}}
\newcommand{\F}{\mathbb{F}}
\newcommand{\E}{\mathcal{E}}

\newcommand{\concept}[1]{\textbf{#1}}
\newcommand{\setl}[1]{\{#1\}}
\newcommand{\setp}[2]{\{#1\mid#2\}}

\setlength{\parskip}{0.5em}


\title{GAAL -- Seção 1 -- Exercício 5}
\author{\empty}
\date{\empty}
\newcommand\onlyinsubfileone\maketitle

\begin{document}

\onlyinsubfileone

\begin{exercicio-gaal}[E5.S1]
  Dados os pontos $A$, $B$ no eixo $E$, defina a distância orientada $\delta(A,B)$ entre eles pondo $\delta(A,B) = d(A,B)$ se $A$ está à esquerda de $B$ e $\delta(A,B) = -d(A,B)$ se $A$ está à direita de $B$.
  Prove que para quaisquer $A$, $B$ e $C$ do eixo $E$ tem-se $\delta(A,B) + \delta(B,C) + \delta (C,A) = 0$.
\end{exercicio-gaal}

\begin{proof}[Solução]
  Sem perda de generalidade podemos supor que $A$ está à esquerda de $B$ e que $B$ está à esquerda de $C$.
  Logo
  \[
    \delta(A,B) + \delta(B,C) + \delta (C,A) = d(A,B) + d(B,C) - d(C,A) = 0
  \]
  pois $d(A,B) + d(B,C) = d(C,A)$, já que o ponto $B$ pertence ao segmento de reta $AC$.
\end{proof}

\end{document}
