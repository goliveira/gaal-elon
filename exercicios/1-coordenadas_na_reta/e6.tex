\documentclass[a4paper,11pt]{article}
\usepackage[brazilian]{babel}
\usepackage[T1]{fontenc}
% \documentclass[a4paper,11pt]{article}
% \usepackage[brazilian]{babel}
% \usepackage[T1]{fontenc}
\usepackage[utf8]{inputenc}
\usepackage{subfiles}
\usepackage{amsmath,amssymb}
\usepackage{amsthm}
\usepackage{commath}
\usepackage{framed}
\usepackage{enumerate}
\usepackage{hyperref}

\usepackage{fancyhdr}
\pagestyle{fancyplain}
\fancyhf{}
\renewcommand{\headrulewidth}{0pt}
\fancyfoot[EOL]{\footnotesize Gustavo de Oliveira}
\fancyfoot[EOC]{\thepage}
\fancyfoot[EOR]{\footnotesize \today}

\newtheorem{teorema}{Teorema}
\newtheorem{theorem}{Theorem}
\theoremstyle{definition}
\newtheorem{exercicio}{Exercício}
\newtheorem{lema}{Lema}
\newtheorem{lemma}{Lemma}

\newcommand{\re}{\mathrm{Re}\,}
\newcommand{\im}{\mathrm{Im}\,}

\newcommand{\N}{\mathbb{N}}
\newcommand{\Z}{\mathbb{Z}}
\newcommand{\R}{\mathbb{R}}
\newcommand{\C}{\mathbb{C}}
\newcommand{\F}{\mathbb{F}}
\newcommand{\E}{\mathcal{E}}

\newcommand{\concept}[1]{\textbf{#1}}
\newcommand{\setl}[1]{\{#1\}}
\newcommand{\setp}[2]{\{#1\mid#2\}}

\setlength{\parskip}{0.5em}


\title{GAAL -- Seção 1 -- Exercício 6}
\author{\empty}
\date{\empty}
\newcommand\onlyinsubfileone\maketitle

\begin{document}

\onlyinsubfileone

\begin{exercicio-gaal}[E6.S1]
  Sejam $a < b < c$ respectivamente as coordenadas dos pontos $A$, $B$ e $C$ situados sobre um eixo.
  Sabendo que $a = 17$, $c = 32$ e
  \[
    \frac{d(A,B)}{d(A,C)} = \frac{2}{3},
  \]
  qual é o valor de $b$?
\end{exercicio-gaal}

\begin{proof}[Solução]
  Usando a fórmula $d(X,Y) = |x-y|$, temos que
  \[
    \frac{3}{2} = \frac{d(A,B)}{d(A,C)} = \frac{|a-b|}{|a-c|} = \frac{|17-b|}{|17-32|}.
  \]
  Como $b > 17$ e $32 > 17$, essa equação é equivalente a
  \[
    \frac{b-17}{32-17} = \frac{2}{3},
  \]
  ou seja,
  \[
    b = \frac{2}{3} 15 + 17 = 27.
  \]
\end{proof}

\end{document}
