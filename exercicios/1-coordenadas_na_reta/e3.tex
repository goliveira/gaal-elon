\documentclass[a4paper,11pt]{article}
\usepackage[brazilian]{babel}
\usepackage[T1]{fontenc}
% \documentclass[a4paper,11pt]{article}
% \usepackage[brazilian]{babel}
% \usepackage[T1]{fontenc}
\usepackage[utf8]{inputenc}
\usepackage{subfiles}
\usepackage{amsmath,amssymb}
\usepackage{amsthm}
\usepackage{commath}
\usepackage{framed}
\usepackage{enumerate}
\usepackage{hyperref}

\usepackage{fancyhdr}
\pagestyle{fancyplain}
\fancyhf{}
\renewcommand{\headrulewidth}{0pt}
\fancyfoot[EOL]{\footnotesize Gustavo de Oliveira}
\fancyfoot[EOC]{\thepage}
\fancyfoot[EOR]{\footnotesize \today}

\newtheorem{teorema}{Teorema}
\newtheorem{theorem}{Theorem}
\theoremstyle{definition}
\newtheorem{exercicio}{Exercício}
\newtheorem{lema}{Lema}
\newtheorem{lemma}{Lemma}

\newcommand{\re}{\mathrm{Re}\,}
\newcommand{\im}{\mathrm{Im}\,}

\newcommand{\N}{\mathbb{N}}
\newcommand{\Z}{\mathbb{Z}}
\newcommand{\R}{\mathbb{R}}
\newcommand{\C}{\mathbb{C}}
\newcommand{\F}{\mathbb{F}}
\newcommand{\E}{\mathcal{E}}

\newcommand{\concept}[1]{\textbf{#1}}
\newcommand{\setl}[1]{\{#1\}}
\newcommand{\setp}[2]{\{#1\mid#2\}}

\setlength{\parskip}{0.5em}


\title{GAAL -- Seção 1 -- Exercício 3}
\author{\empty}
\date{\empty}
\newcommand\onlyinsubfileone\maketitle

\begin{document}

\onlyinsubfileone

\begin{exercicio-gaal}[E3.S1]
  Se $O$ é a origem do eixo $E$ e $A$ é o ponto desse eixo que tem coordenada $1$, qual é a coordenada do ponto $X$ que divide o segmento de reta $OA$ em média e extrema razão?
  No Exercício 2, calcule a \emph{razão áurea} $d(O,X)/d(O,A)$.
\end{exercicio-gaal}

\begin{proof}[Solução]
  O ponto $X$ tem coordenada
  \[
    x = \frac{1}{2} ( (3-\sqrt{5}) 0 + (\sqrt{5}-1) 1 ) = \frac{\sqrt{5}-1}{2}.
  \]
  Calculamos $d(O,A) = |0-1| = 1$.
  Portanto a razão áurea é
  \[
    \frac{d(O,X)}{d(O,A)} = \frac{\sqrt{5}-1}{2}.
  \]
\end{proof}

\end{document}
