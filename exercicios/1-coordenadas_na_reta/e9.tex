\documentclass[a4paper,11pt]{article}
\usepackage[brazilian]{babel}
\usepackage[T1]{fontenc}
% \documentclass[a4paper,11pt]{article}
% \usepackage[brazilian]{babel}
% \usepackage[T1]{fontenc}
\usepackage[utf8]{inputenc}
\usepackage{subfiles}
\usepackage{amsmath,amssymb}
\usepackage{amsthm}
\usepackage{commath}
\usepackage{framed}
\usepackage{enumerate}
\usepackage{hyperref}

\usepackage{fancyhdr}
\pagestyle{fancyplain}
\fancyhf{}
\renewcommand{\headrulewidth}{0pt}
\fancyfoot[EOL]{\footnotesize Gustavo de Oliveira}
\fancyfoot[EOC]{\thepage}
\fancyfoot[EOR]{\footnotesize \today}

\newtheorem{teorema}{Teorema}
\newtheorem{theorem}{Theorem}
\theoremstyle{definition}
\newtheorem{exercicio}{Exercício}
\newtheorem{lema}{Lema}
\newtheorem{lemma}{Lemma}

\newcommand{\re}{\mathrm{Re}\,}
\newcommand{\im}{\mathrm{Im}\,}

\newcommand{\N}{\mathbb{N}}
\newcommand{\Z}{\mathbb{Z}}
\newcommand{\R}{\mathbb{R}}
\newcommand{\C}{\mathbb{C}}
\newcommand{\F}{\mathbb{F}}
\newcommand{\E}{\mathcal{E}}

\newcommand{\concept}[1]{\textbf{#1}}
\newcommand{\setl}[1]{\{#1\}}
\newcommand{\setp}[2]{\{#1\mid#2\}}

\setlength{\parskip}{0.5em}


\title{GAAL -- Seção 1 -- Exercício 9}
\author{\empty}
\date{\empty}
\newcommand\onlyinsubfileone\maketitle

\begin{document}

\onlyinsubfileone

\begin{exercicio-gaal}[E9.S1]
  Seja $f : \R \to \R$ uma função tal que $|f(x)-f(y)| = |x-y|$ para quaisquer $x,y \in \R$.
  \begin{enumerate}[(i)]
    \item
      Pondo $f(0) = a$, defina a função $g : \R \to \R$ assim: $g(x) = f(x) - a$.
      Prove então que $|g(x)| = |x|$ para todo $x \in \R$.
      Em particular, $g(1) = 1$ ou $g(1) = -1$.
      Também $(g(x))^2 = x^2$.
    \item
      Use a identidade $xy = \frac{1}{2}[ x^2 + y^2 - (x-y)^2 ]$ para provar a igualdade $xy = g(x)\,g(y)$.
    \item
      Se $g(1) = 1$, mostre que $g(x) = x$ para todo $x \in \R$.
      Se $g(1) = -1$, mostre que $g(x) = -x$ para todo $x$.
    \item
      Conclua que $f(x) = x + a$ para todo $x \in \R$ ou então $f(x) = -x + a$ para todo $x$.
  \end{enumerate}
\end{exercicio-gaal}

\begin{proof}[Solução]
  (i)
  Observamos que $g(x) = f(x) - a = f(x) - f(0)$.
  Logo
  \[
    |g(x)| = |f(x)-f(0)| = |x-0| = |x|
  \]
  para todo $x \in \R$.
  Sendo assim, $|g(1)| = 1$, o que implica $g(1) = 1$ ou $g(1) = -1$.
  Temos também que
  \[
    g(x)^2 = |g(x)|^2 = |x|^2 = x^2.
  \]

  (ii)
  Usando as definições e propriedades, calculamos
  \begin{align*}
    xy & = 2^{-1} [x^2 + y^2 - (x-y)^2 ] \\
    & = 2^{-1} [ x^2 + y^2 - |x-y|^2 ] \\
    & = 2^{-1} [ g(x)^2 + g(y)^2 - |f(x) - f(y)|^2 ] \\
    & = 2^{-1} [ g(x)^2 + g(y)^2 - |f(x) - a - f(y) + a|^2 ] \\
    & = 2^{-1} [ g(x)^2 + g(y)^2 - |g(x) - g(y)|^2 ] \\
    & = 2^{-1} [ g(x)^2 + g(y)^2 - (g(x) - g(y))^2 ] \\
    & = g(x)g(y).
  \end{align*}

  (iii)
  Se $g(1)=1$, então $x = x(1) = g(x)g(1) = g(x)$ para todo $x \in \R$.
  Se $g(1)=-1$, então $x = x(1) = g(x)g(1) = g(x)(-1) = -g(x)$ para todo $x \in \R$.
  Portanto $g(x) = -x$ para todo $x$.

  (iv)
  Observamos que $f(x) = g(x) + a$.
  Pela parte $(i)$, temos $g(1) = 1$ ou $g(1) = -1$.
  Usando (iii), isso implica $g(x)=x$ ou $g(x) = -x$ para todo $x$, respectivamente.
  Portanto $f(x) = x + a$ ou $f(x) = -x + a$ para todo $x \in \R$.
\end{proof}

\end{document}
