\documentclass[a4paper,11pt]{article}
\usepackage[brazilian]{babel}
\usepackage[T1]{fontenc}
% \documentclass[a4paper,11pt]{article}
% \usepackage[brazilian]{babel}
% \usepackage[T1]{fontenc}
\usepackage[utf8]{inputenc}
\usepackage{subfiles}
\usepackage{amsmath,amssymb}
\usepackage{amsthm}
\usepackage{commath}
\usepackage{framed}
\usepackage{enumerate}
\usepackage{hyperref}

\usepackage{fancyhdr}
\pagestyle{fancyplain}
\fancyhf{}
\renewcommand{\headrulewidth}{0pt}
\fancyfoot[EOL]{\footnotesize Gustavo de Oliveira}
\fancyfoot[EOC]{\thepage}
\fancyfoot[EOR]{\footnotesize \today}

\newtheorem{teorema}{Teorema}
\newtheorem{theorem}{Theorem}
\theoremstyle{definition}
\newtheorem{exercicio}{Exercício}
\newtheorem{lema}{Lema}
\newtheorem{lemma}{Lemma}

\newcommand{\re}{\mathrm{Re}\,}
\newcommand{\im}{\mathrm{Im}\,}

\newcommand{\N}{\mathbb{N}}
\newcommand{\Z}{\mathbb{Z}}
\newcommand{\R}{\mathbb{R}}
\newcommand{\C}{\mathbb{C}}
\newcommand{\F}{\mathbb{F}}
\newcommand{\E}{\mathcal{E}}

\newcommand{\concept}[1]{\textbf{#1}}
\newcommand{\setl}[1]{\{#1\}}
\newcommand{\setp}[2]{\{#1\mid#2\}}

\setlength{\parskip}{0.5em}


\title{GAAL -- Seção 1 -- Exercício 4}
\author{\empty}
\date{\empty}
\newcommand\onlyinsubfileone\maketitle

\begin{document}

\onlyinsubfileone

\begin{exercicio-gaal}[E4.S1]
  Os pontos $A$, $B$ e $X$ sobre o eixo $E$ têm coordenadas $a$, $b$ e $x$ respectivamente.
  Se $X'$ é o simétrico de $X$ em relação ao ponto $A$ e $X''$ é o simétrico de $X'$ em relação a $B$, quais são as coordenadas de $X'$ e $X''$?
\end{exercicio-gaal}

\begin{proof}[Solução]
  Sejam $x'$ e $x''$ as coordenadas de $X'$ e $X''$.
  Como $A$ é o ponto médio de $XX'$, temos $a = (x+x')/2$.
  Logo $x' = 2a - x$.
  Como $B$ é o ponto médio de $X'X''$, temos $b = (x'+x'')/2$.
  Portanto $x'' = 2b - x' = 2(b-a) + x$.
\end{proof}

\end{document}
