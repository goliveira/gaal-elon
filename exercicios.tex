\documentclass[a4paper,11pt]{article}
\usepackage[brazilian]{babel}
\usepackage[utf8]{inputenc}
\usepackage[T1]{fontenc}
\usepackage{amsmath}
\usepackage{amssymb}
\usepackage{amsthm}
\usepackage{indentfirst}

\newcommand{\R}{\mathbb{R}}

\title{Exercícios Resolvidos do Livro\\Geometria Analítica e Álgebra Linear\\de Elon Lages Lima\\(Segunda Edição--Oitava Impressão)}
\author{Gustavo de Oliveira}

\begin{document}

\maketitle

\section*{Seção 14 -- Vetores no Plano}

\textbf{2.}
Prove geometricamente que um quadrilátero é um paralelogramo se, e somente se, suas diagonais se cortam mutuamente ao meio.

\vspace{\baselineskip}

\emph{Solução.}
($\Rightarrow$)
Suponha que o quadrilátero $ABCD$ é um paralelogramo.
O paralelogramo é formado por dois pares de lados.
Em cada par, os lados são paralelos e têm o mesmo comprimento.
Portanto $\overrightarrow{AD} = \overrightarrow{BC}$ e $\overrightarrow{AB} = \overrightarrow{DC}$.
Seja $P$ o ponto médio de $DB$, e seja $Q$ o ponto médio de $AC$.
Vamos provar que $Q = P$.

Escolhemos um sistema de coordenadas $OXY$ de modo que $A = (0,0)$, $B = (b,0)$ e $D = (c, d)$.
Logo $\overrightarrow{AD} = (c,d)$ e $C = B + \overrightarrow{AD} = (b + c, d)$.
Calculando os pontos $P$ e $Q$, obtemos
\begin{align*}
  P & = \left( \frac{c+b}{2}, \frac{d+0}{2} \right) = \left( \frac{b+c}{2}, \frac{d}{2} \right), \\
  Q & = \left( \frac{b + c + 0}{2}, \frac{d+0}{2} \right) = \left( \frac{b+c}{2}, \frac{d}{2} \right).
\end{align*}
Portanto $P = Q$.

($\Leftarrow$)
Seja $P$ o ponto médio de $DB$, e seja $Q$ o ponto médio de $AC$.
Suponha que as diagonais do paralelogramo se cortam mutuamente ao meio, ou seja, suponha que $P = Q$.
Escolhemos um sistema de coordenadas $OXY$ de modo que $A = (0,0)$, $B = (b,0)$ e $D = (c, d)$.
Temos então $\overrightarrow{AD} = (c,d)$ e $\overrightarrow{AB} = (b,0)$.
Escrevemos $C = (x,y)$.
Vamos determinar $x$ e $y$.
Calculando os pontos $P$ e $Q$, obtemos
\begin{align*}
  P & = \left( \frac{c+b}{2}, \frac{d}{2} \right), \\
  Q & = \left( \frac{x}{2}, \frac{y}{2} \right).
\end{align*}
A igualdade $P = Q$ implica $x = c + b$ e $y = d$.
Logo $(x,y) = (b + c, d)$, ou seja, $C = (b + c, d)$.
Portanto $C = B + \overrightarrow{AD}$ e $C = D + \overrightarrow{AB}$, ou seja, $\overrightarrow{BC} = \overrightarrow{AD}$ e $\overrightarrow{DC} = \overrightarrow{AB}$.
Isso implica que $ABCD$ é um paralelogramo.

\section*{Seção 15 -- Operações com Vetores}

\textbf{7.}
Seja $P$ um ponto interior ao triângulo $ABC$ tal que $\overrightarrow{PA} \!+\! \overrightarrow{PB} + \overrightarrow{PC} = 0$.
Prove que as retas $AP$, $BP$ e $CP$ são medianas de $ABC$, logo $P$ é o ba\-ri\-cen\-tro desse triângulo.

\vspace{\baselineskip}

\emph{Solução.}
Seja $Q$ o ponto de intersecção da reta $BP$ com o segmento $AC$.
Observamos que $\overrightarrow{QA} = \alpha \overrightarrow{CA}$ para $\alpha \in \R$.
Logo
\[
  \overrightarrow{QC} = \overrightarrow{QA} + \overrightarrow{AC} = \alpha \overrightarrow{CA} - \overrightarrow{CA} = (\alpha - 1) \overrightarrow{CA}.
\]
Vamos provar que $Q$ é o ponto médio do lado $AC$, ou seja, vamos provar que $\alpha = 1/2$.

Escrevemos
\begin{align*}
  \overrightarrow{PA} & = \overrightarrow{PQ} + \overrightarrow{QA} = \overrightarrow{PQ} + \alpha \overrightarrow{CA}, \\
  \overrightarrow{PB} & = \overrightarrow{PQ} + \overrightarrow{QC} + \overrightarrow{CB} = \overrightarrow{PQ} + (\alpha - 1) \overrightarrow{CA} + \overrightarrow{CB}, \\
  \overrightarrow{PC} & = \overrightarrow{PQ} + \overrightarrow{QC} = \overrightarrow{PQ} + (\alpha - 1) \overrightarrow{CA}.
\end{align*}
Logo
\[
  \overrightarrow{PA} + \overrightarrow{PB} + \overrightarrow{PC} = 3 \overrightarrow{PQ} + (3\alpha - 2) \overrightarrow{CA} + \overrightarrow{CB}.
\]
Além disso
\[
  \overrightarrow{BQ} = \overrightarrow{BC} + \overrightarrow{CQ} = - \overrightarrow{CB} - \overrightarrow{QC} = - \overrightarrow{CB} + (1 - \alpha) \overrightarrow{CA}
\]
e
\[
  \overrightarrow{PQ} = \beta \overrightarrow{BQ}
\]
para $\beta \in \R$.
Portanto
\[
  \overrightarrow{PQ} = \beta \overrightarrow{BQ} = - \beta \overrightarrow{CB} + \beta (1 - \alpha) \overrightarrow{CA}.
\]
Consequentemente
\[
  \overrightarrow{PA} + \overrightarrow{PB} + \overrightarrow{PC} = (3 \beta (1 - \alpha) + 3\alpha - 2) \overrightarrow{CA} + (1 - 3 \beta) \overrightarrow{CB}.
\]
Por outro lado, temos $\overrightarrow{PA} + \overrightarrow{PB} + \overrightarrow{PC} = 0$.
Logo
\[
  (3 \beta (1 - \alpha) + 3\alpha - 2) \overrightarrow{CA} + (1 - 3 \beta) \overrightarrow{CB} = 0.
\]
Como $\overrightarrow{CA}$ e $\overrightarrow{CB}$ são linearmente independentes, essa igualdade implica
(veja o Exercício 1 da Seção 15)
\[
  (3 \beta(1 - \alpha) + 3\alpha - 2) = 0 \qquad \text{e} \qquad 1 - 3 \beta = 0.
\]
A segunda equação implica $\beta = 1/3$.
Substituindo esse valor de $\beta$ na primeira equação, obtemos $3(1/3)(1 - \alpha) + 3\alpha - 2 = 0$, ou seja $\alpha = 1/2$.
Portanto $Q$ é o ponto médio de $AC$.
Renomeando os pontos, obtemos a demonstração para as medianas correspondentes aos outros vértices do triângulo.

\section*{Seção 16 -- Equação da Elipse}

\textbf{10.}
Quais são as tangentes à elipse $x^2 + 4y^2 = 32$ que têm inclinação igual a $1/2$?

\vspace{\baselineskip}

\emph{Solução.}
Uma reta com inclinação igual a $1/2$ é dada por $y = (1/2)x + b$ para $b \in \R$.
Vamos determinar $b$ para o qual a reta $y = (1/2)x + b$ é tangente à elipse $x^2 + 4y^2 = 32$, ou seja, vamos determinar $b$ para o qual o sistema
\begin{align*}
  & x^2 + 4y^2 = 32, \\
  & y = (1/2)x + b
\end{align*}
tem apenas uma solução.
Substituindo a segunda equação na primeira e desenvolvendo obtemos
\[
  2x^2 + 4bx + (4b^2 - 32) = 0.
\]
Essa equação em $x$ possui apenas uma solução se e somente se o discriminante da equação é igual a zero, ou seja,
\[
  \Delta = -16b^2 + 16^2 = 0.
\]
Isso implica em $b = \pm 4$.
Portanto as retas tangentes são
\[
  y = \frac{1}{2} x - 4 \qquad \text{e} \qquad y = \frac{1}{2} x + 4.
\]

\end{document}
