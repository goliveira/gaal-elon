\documentclass[a4paper,11pt]{article}
\usepackage[brazilian]{babel}
\usepackage[utf8]{inputenc}
\usepackage[T1]{fontenc}
\usepackage{amsmath}
\usepackage{amssymb}
\usepackage{amsthm}
\usepackage{framed}
\usepackage{indentfirst}

% \usepackage{titlesec}
% \newcommand{\sectionbreak}{\clearpage}

% \newcommand{\N}{\mathbb{N}}
% \newcommand{\Z}{\mathbb{Z}}
% \newcommand{\Q}{\mathbb{Q}}
\newcommand{\R}{\mathbb{R}}
% \newcommand{\C}{\mathbb{C}}
% \newcommand{\F}{\mathbb{F}}

% \newtheorem{theorem}{Teorema}
% \theoremstyle{definition}
% \newtheorem{example}{Exemplo}

% \newcommand{\setl}[1]{\{#1\}}
% \newcommand{\set}[2]{\{#1\mid#2\}}
% \newcommand{\concept}[1]{\emph{#1}}
% \newcommand{\topic}[1]{\subsection*{#1}}

\title{Exercícios Resolvidos do Livro\\ Geometria Analítica e Álgebra Linear\\ de Elon Lages Lima\\ (Segunda Edição--Oitava Impressão)}
\author{Gustavo de Oliveira}

\begin{document}

\maketitle

\section{Seção 14 -- Vetores no Plano}

\textbf{2.}
Prove geometricamente que um quadrilátero é um paralelogramo se, e somente se, suas diagonais se cortam mutuamente ao meio.

\textbf{Solução.}
($\Rightarrow$)
Suponha que o quadrilátero $ABCD$ é um paralelogramo.
O paralelogramo é formado por dois pares de lados.
Em cada par, os lados são paralelos e têm o mesmo comprimento.
Portanto $\overrightarrow{AD} = \overrightarrow{BC}$ e $\overrightarrow{AB} = \overrightarrow{DC}$.
Seja $P$ o ponto médio de $DB$, e seja $Q$ o ponto médio de $AC$.
Queremos provar que $Q = P$.

Escolhemos um sistema de coordenadas $OXY$ de modo que $A = (0,0)$, $B = (b,0)$ e $D = (c, d)$.
Logo $\overrightarrow{AD} = (c,d)$ e $C = B + \overrightarrow{AD} = (b + c, d)$.
Calculando os pontos $P$ e $Q$, obtemos
\begin{align*}
  P & = \left( \frac{c+b}{2}, \frac{d+0}{2} \right) = \left( \frac{b+c}{2}, \frac{d}{2} \right), \\
  Q & = \left( \frac{b + c + 0}{2}, \frac{d+0}{2} \right) = \left( \frac{b+c}{2}, \frac{d}{2} \right).
\end{align*}
Portanto $P = Q$.

($\Leftarrow$)
Seja $P$ o ponto médio de $DB$, e seja $Q$ o ponto médio de $AC$.
Suponha que as diagonais do paralelogramo se cortam mutuamente ao meio, ou seja, suponha que $P = Q$.
Escolhemos um sistema de coordenadas $OXY$ de modo que $A = (0,0)$, $B = (b,0)$ e $D = (c, d)$.
Temos então $\overrightarrow{AD} = (c,d)$ e $\overrightarrow{AB} = (b,0)$.
Escrevemos $C = (x,y)$.
Vamos determinar $x$ e $y$.
Como $P$ e $Q$ são pontos médios, obtemos
\begin{align*}
  P & = \left( \frac{c+b}{2}, \frac{d}{2} \right), \\
  Q & = \left( \frac{x}{2}, \frac{y}{2} \right).
\end{align*}
A igualdade $P = Q$ implica $x = c + b$ e $y = d$.
Logo $(x,y) = (b + c, d)$, ou seja, $C = (b + c, d)$.
Portanto $C = B + \overrightarrow{AD}$ e $C = D + \overrightarrow{AB}$, ou seja, $\overrightarrow{BC} = \overrightarrow{AD}$ e $\overrightarrow{DC} = \overrightarrow{AB}$.
Isso implica que $ABCD$ é um paralelogramo.

\end{document}
