\documentclass[a4paper,11pt]{article}
\usepackage[brazilian]{babel}
\usepackage[utf8]{inputenc}
\usepackage[T1]{fontenc}
\usepackage{amsmath}
\usepackage{amssymb}
\usepackage{amsthm}
\usepackage{indentfirst}
\usepackage{enumerate}

\newcommand{\R}{\mathbb{R}}

\title{Exercícios Resolvidos do Livro\\Geometria Analítica e Álgebra Linear\\de Elon Lages Lima\\(Segunda Edição--Oitava Impressão)}
\author{Gustavo de Oliveira}

\begin{document}

\maketitle

\section*{Seção 1 -- Coordenadas na Reta}

\textbf{1.}
Sejam $a < b$ respectivamente as coordenadas dos pontos $A$ e $B$ sobre o eixo $E$.
Determine as coordenadas dos pontos $X_1, \dots, X_{n-1}$ que dividem o segmento $AB$ em $n$ partes iguais.

\vspace{\baselineskip}

\emph{Solução.}
Para $j = 1, \dots, n-1$, observamos que $d(X_j,a) = j \, d(A,B)/n$.
Seja $x_j$ a coordenada do ponto $X_j$.
Então $|x_j - a| = j|a-b|/n$, ou seja, $x_j - a = j(b-a)/n$, pois $x_j > a$ e $b > a$.
Portanto $x_j = a + j(b-a)/n$.

\section*{Seção 14 -- Vetores no Plano}

\textbf{2.}
Prove geometricamente que um quadrilátero é um paralelogramo se, e somente se, suas diagonais se cortam mutuamente ao meio.

\vspace{\baselineskip}

\emph{Solução.}
($\Rightarrow$)
Suponha que o quadrilátero $ABCD$ é um paralelogramo.
O paralelogramo é formado por dois pares de lados.
Em cada par de lados, os lados são paralelos e têm o mesmo comprimento.
Portanto $\overrightarrow{AD} = \overrightarrow{BC}$ e $\overrightarrow{AB} = \overrightarrow{DC}$.
Seja $P$ o ponto médio de $DB$, e seja $Q$ o ponto médio de $AC$.
Vamos provar que $Q = P$.
Escolhemos um sistema de coordenadas $OXY$ de modo que $A = (0,0)$, $B = (b,0)$ e $D = (c, d)$.
Logo $\overrightarrow{AD} = (c,d)$ e $C = B + \overrightarrow{AD} = (b + c, d)$.
Calculando os pontos $P$ e $Q$, obtemos
\begin{align*}
  P & = \left( \frac{c+b}{2}, \frac{d+0}{2} \right) = \left( \frac{b+c}{2}, \frac{d}{2} \right), \\
  Q & = \left( \frac{b + c + 0}{2}, \frac{d+0}{2} \right) = \left( \frac{b+c}{2}, \frac{d}{2} \right).
\end{align*}
Portanto $P = Q$.

($\Leftarrow$)
Seja $P$ o ponto médio de $DB$, e seja $Q$ o ponto médio de $AC$.
Suponha que as diagonais do paralelogramo se cortam mutuamente ao meio, ou seja, suponha que $P = Q$.
Escolhemos um sistema de coordenadas $OXY$ de modo que $A = (0,0)$, $B = (b,0)$ e $D = (c, d)$.
Temos então $\overrightarrow{AD} = (c,d)$ e $\overrightarrow{AB} = (b,0)$.
Escrevemos $C = (x,y)$.
Vamos determinar $x$ e $y$.
Calculando os pontos $P$ e $Q$, obtemos
\begin{align*}
  P & = \left( \frac{c+b}{2}, \frac{d}{2} \right), \\
  Q & = \left( \frac{x}{2}, \frac{y}{2} \right).
\end{align*}
A igualdade $P = Q$ implica $x = c + b$ e $y = d$.
Logo $(x,y) = (b + c, d)$, ou seja, $C = (b + c, d)$.
Portanto $C = B + \overrightarrow{AD}$ e $C = D + \overrightarrow{AB}$, ou seja, $\overrightarrow{BC} = \overrightarrow{AD}$ e $\overrightarrow{DC} = \overrightarrow{AB}$.
Isso implica que $ABCD$ é um paralelogramo.

\section*{Seção 15 -- Operações com Vetores}

\textbf{7.}
Seja $P$ um ponto interior ao triângulo $ABC$ tal que $\overrightarrow{PA} \!+\! \overrightarrow{PB} + \overrightarrow{PC} = 0$.
Prove que as retas $AP$, $BP$ e $CP$ são medianas de $ABC$, logo $P$ é o ba\-ri\-cen\-tro desse triângulo.

\vspace{\baselineskip}

\emph{Solução.}
Seja $Q$ o ponto de intersecção da reta $BP$ com o segmento $AC$.
Observamos que $\overrightarrow{QA} = \alpha \overrightarrow{CA}$ para $\alpha \in \R$.
Logo
\[
  \overrightarrow{QC} = \overrightarrow{QA} + \overrightarrow{AC} = \alpha \overrightarrow{CA} - \overrightarrow{CA} = (\alpha - 1) \overrightarrow{CA}.
\]
Vamos provar que $Q$ é o ponto médio do lado $AC$, ou seja, vamos provar que $\alpha = 1/2$.
Escrevemos
\begin{align*}
  \overrightarrow{PA} & = \overrightarrow{PQ} + \overrightarrow{QA} = \overrightarrow{PQ} + \alpha \overrightarrow{CA}, \\
  \overrightarrow{PB} & = \overrightarrow{PQ} + \overrightarrow{QC} + \overrightarrow{CB} = \overrightarrow{PQ} + (\alpha - 1) \overrightarrow{CA} + \overrightarrow{CB}, \\
  \overrightarrow{PC} & = \overrightarrow{PQ} + \overrightarrow{QC} = \overrightarrow{PQ} + (\alpha - 1) \overrightarrow{CA}.
\end{align*}
Logo
\[
  \overrightarrow{PA} + \overrightarrow{PB} + \overrightarrow{PC} = 3 \overrightarrow{PQ} + (3\alpha - 2) \overrightarrow{CA} + \overrightarrow{CB}.
\]
Além disso,
\[
  \overrightarrow{BQ} = \overrightarrow{BC} + \overrightarrow{CQ} = - \overrightarrow{CB} - \overrightarrow{QC} = - \overrightarrow{CB} + (1 - \alpha) \overrightarrow{CA}
\]
e, para algum $\beta \in \R$, temos
\[
  \overrightarrow{PQ} = \beta \overrightarrow{BQ}.
\]
Portanto
\[
  \overrightarrow{PQ} = \beta \overrightarrow{BQ} = - \beta \overrightarrow{CB} + \beta (1 - \alpha) \overrightarrow{CA}.
\]
Consequentemente
\[
  \overrightarrow{PA} + \overrightarrow{PB} + \overrightarrow{PC} = (3 \beta (1 - \alpha) + 3\alpha - 2) \overrightarrow{CA} + (1 - 3 \beta) \overrightarrow{CB}.
\]
Por outro lado, temos $\overrightarrow{PA} + \overrightarrow{PB} + \overrightarrow{PC} = 0$.
Logo
\[
  (3 \beta (1 - \alpha) + 3\alpha - 2) \overrightarrow{CA} + (1 - 3 \beta) \overrightarrow{CB} = 0.
\]
Como $\overrightarrow{CA}$ e $\overrightarrow{CB}$ são linearmente independentes, essa igualdade implica
(veja o Exercício 1 da Seção 15)
\[
  (3 \beta(1 - \alpha) + 3\alpha - 2) = 0 \qquad \text{e} \qquad 1 - 3 \beta = 0.
\]
A segunda equação implica $\beta = 1/3$.
Substituindo esse valor de $\beta$ na primeira equação, obtemos $3(1/3)(1 - \alpha) + 3\alpha - 2 = 0$, ou seja, $\alpha = 1/2$.
Portanto $Q$ é o ponto médio de $AC$.
Renomeando os pontos, obtemos a demonstração para as medianas correspondentes aos outros vértices do triângulo.

\section*{Seção 16 -- Equação da Elipse}

\textbf{10.}
Quais são as tangentes à elipse $x^2 + 4y^2 = 32$ que têm inclinação igual a $1/2$?

\vspace{\baselineskip}

\emph{Solução.}
Uma reta com inclinação $1/2$ é dada por $y = (1/2)x + b$ para $b \in \R$.
Vamos determinar os valores de $b$ para os quais a reta $y = (1/2)x + b$ é tangente à elipse $x^2 + 4y^2 = 32$, ou seja, vamos determinar os valores de $b$ para os quais o sistema
\begin{align*}
  & x^2 + 4y^2 = 32 \\
  & y = (1/2)x + b
\end{align*}
tem apenas uma solução.
Substituindo a segunda equação na primeira e desenvolvendo, obtemos
\[
  2x^2 + 4bx + (4b^2 - 32) = 0.
\]
Essa equação possui apenas uma solução se, e somente se, o discriminante da equação é igual a zero, ou seja,
\[
  \Delta = -16b^2 + 16^2 = 0.
\]
Isso implica $b = \pm 4$.
Portanto, as retas tangentes são
\[
  y = \frac{1}{2} x - 4 \qquad \text{e} \qquad y = \frac{1}{2} x + 4.
\]

\section*{Seção 17 -- Equação da Hipérbole}

\textbf{2.}
Para todo ponto $P = (m,n)$ na hipérbole $H : x^2/a^2 - y^2/b^2 = 1$, mostre que a reta $r: (m/a^2)x - (n/b^2)y = 1$ tem apenas o ponto $P$ em comum com $H$.
A reta $r$ chama-se a \emph{tangente} a $H$ no ponto $P$.

\vspace{\baselineskip}

\emph{Solução.}
A reta $r$ é tangente à hipérbole $H$ no ponto $P$ se, e somente se, $x = m$ e $y = n$ é a única solução do sistema
\begin{align*}
  (m/a^2) x - (n/b^2) y & = 1 \\
  x^2/a^2 - y^2/b^2 & = 1.
\end{align*}
A primeira equação implica
\[
  x = \frac{a^2}{m} \left( 1 + \frac{n}{b^2} \right).
\]
Substituindo essa expressão para $x$ na segunda equação e desenvolvendo, obtemos
\[
  (a^2 n^2 - b^2 m^2) y^2 + b^2 a^2 2ny + b^4(a^2 - m^2) = 0.
\]
Como $P$ pertence à hipérbole, temos $a^2 n^2 - b^2 m^2 = -a^2 b^2$.
Substituindo essa expressão na equação anterior e simplificado, encontramos
\[
  -a^2 y^2 + a^2 2ny + b^2 (a^2 - m^2) = 0.
\]
Calculando o discriminante $\Delta$ dessa equação quadrática, obtemos
\[
  \Delta = 4 a^2 ( a^2 n^2 - b^2 m^2 + b^2 a^2 ) = 4 a^2 (-a^2 b^2 + b^2 a^2 ) = 4 a^2 (0) = 0.
\]
Nesse cálculo, usamos novamente que $P$ pertence a $H$.
Como $\Delta = 0$, a equação para $y$ possui apenas uma solução.
Associado a essa solução temos apenas um valor para $x$.
Portanto o sistema de equações possui apenas uma solução $(x,y)$, ou seja, a reta $r$ é tangente à hipérbole $H$.

\section*{Seção 20 -- Formas Quadráticas}

\textbf{1.}
Para cada uma das formas quadráticas abaixo, execute as seguintes tarefas:
\begin{enumerate}
  \item
    Escreva sua matriz e sua equação característica;
  \item
    Obtenha seus autovalores;
  \item
    Descreva suas linhas de nível;
  \item
    Ache autovetores unitários ortogonais $u$ e $u^*$;
  \item
    Determine os novos eixos em cujas coordenadas a forma quadrática se exprime como $A' s^2 + C' t^2$;
  \item
    Ache os focos da cônica $A' s^2 + C' t^2 = 1$ em termos das coordenadas $x$~e~$y$.
\end{enumerate}
As formas quadráticas são:
\begin{enumerate}[(a)]
  \item
    $\varphi(x,y) = x^2 + xy + y^2$.
  \item
    $\varphi(x,y) = xy$.
  \item
    $\varphi(x,y) = x^2 -6 xy +9 y^2$.
  \item
    $\varphi(x,y) = x^2 + xy - y^2$.
  \item
    $\varphi(x,y) = x^2 +2 xy -3 y^2$.
  \item
    $\varphi(x,y) = x^2 +24 xy -6 y^2$.
\end{enumerate}

\vspace{\baselineskip}

\emph{Solução.}
(a)
A matriz de $\varphi$ é
\[
  \begin{bmatrix}
    1 & 1/2 \\
    1/2 & 1
  \end{bmatrix}.
\]
A equação característica de $\varphi$ é $\lambda^2 - 2\lambda + 3/4 = 0$.
Logo os autovalores são $\lambda_1 = 3/2$ e $\lambda_2 = 1/2$.
Os autovetores unitários correspondentes são
\[
  u = \left( \frac{1}{\sqrt{2}}, \frac{1}{\sqrt{2}} \right), \qquad u^* = \left( \frac{-1}{\sqrt{2}}, \frac{1}{\sqrt{2}} \right).
\]
Consequentemente, se efetuarmos a mudança de variáveis
\begin{align*}
  x & = \frac{1}{\sqrt{2}} s - \frac{1}{\sqrt{2}} t, \\
  y & = \frac{1}{\sqrt{2}} s + \frac{1}{\sqrt{2}} t,
\end{align*}
a forma quadrática assume a forma
\[
  \overline{\varphi}(s,t) = \frac{3}{2} s^2 + \frac{1}{2} t^2 = \frac{s^2}{2/3} + \frac{t^2}{2}.
\]
Para $d < 0$, as linhas de nível $\overline{\varphi}(s,t) = d$ são o conjunto vazio.
Para $d = 0$, a linha de nível $\overline{\varphi}(s,t) = d$ é o ponto $(0,0)$.
Para $d > 0$, as linhas de nível $\overline{\varphi}(s,t) = d$ são elipses.
Nesse caso, temos uma elipse com $c^2 = a^2 + b^2$, ou seja, $c = 2 \sqrt{2d/3}$, e portanto os focos da elipse são $(-c,0)$ e $(c,0)$, no sistema $s$ e $t$.
Em termos das coordenadas $x$ e $y$, os focos são, respectivamente,
\[
  \left( -\frac{2\sqrt{d}}{\sqrt{3}}, -\frac{2\sqrt{d}}{\sqrt{3}} \right), \qquad \left( \frac{2\sqrt{d}}{\sqrt{3}}, \frac{2\sqrt{d}}{\sqrt{3}} \right).
\]

(c)
A matriz de $\varphi$ é
\[
  \begin{bmatrix}
    1 & -3 \\
    -3 & 9
  \end{bmatrix}.
\]
A equação característica de $\varphi$ é $\lambda^2 - 10\lambda = 0$.
Logo os autovalores são $\lambda_1 = 0$ e $\lambda_2 = 10$.
Os autovetores unitários correspondentes são
\[
  u = \left( \frac{3}{\sqrt{10}}, \frac{1}{\sqrt{10}} \right), \qquad u^* = \left( -\frac{1}{\sqrt{10}}, \frac{3}{\sqrt{10}} \right).
\]
Consequentemente, se efetuarmos a mudança de variáveis
\begin{align*}
  x & = \frac{3}{\sqrt{10}} s - \frac{1}{\sqrt{10}} t, \\
  y & = \frac{1}{\sqrt{10}} s + \frac{3}{\sqrt{10}} t,
\end{align*}
a forma quadrática assume a forma
\[
  \overline{\varphi}(s,t) = 10 t^2.
\]
Para $d < 0$, as linhas de nível $\overline{\varphi}(s,t) = d$ são o conjunto vazio.
Para $d = 0$, a linha de nível $\overline{\varphi}(s,t) = d$ é a reta horizontal $t = 0$ que passa pela origem.
Para $d > 0$, as linhas de nível $\overline{\varphi}(s,t) = d$ são o par de retas horizontais $t = \pm \sqrt{d/10}$.
Em termos das coordenadas $x$ e $y$, a reta $t = 0$ é $x - 3y = 0$, e as retas $t = \pm \sqrt{d/10}$ são $x - 3y = \mp \sqrt{d}$.

(d)
A matriz de $\varphi$ é
\[
  \begin{bmatrix}
    1 & 1/2 \\
    1/2 & -1
  \end{bmatrix}.
\]
A equação característica de $\varphi$ é $\lambda^2 - 5/4 = 0$.
Logo os autovalores são $\lambda_1 = \sqrt{5}/2$ e $\lambda_2 = -\sqrt{5}/2$.
Os autovetores unitários correspondentes são
\[
  u = \left( \frac{1}{\sqrt{10 - 4\sqrt{5}}}, \frac{\sqrt{5}-2}{\sqrt{10 - 4\sqrt{5}}} \right), \qquad u^* = \left( \frac{2-\sqrt{5}}{\sqrt{10 - 4\sqrt{5}}}, \frac{1}{\sqrt{10 - 4\sqrt{5}}} \right).
\]
Consequentemente, se efetuarmos a mudança de variáveis
\begin{align*}
  x & = u_1 s - u_2 t, \\
  y & = u_2 s + u_1 t,
\end{align*}
onde $u_1$ e $u_2$ são as coordenadas de $u$, a forma quadrática assume a forma
\[
  \overline{\varphi}(s,t) = \frac{\sqrt{5}}{2} s^2 - \frac{\sqrt{5}}{2} t^2 = \frac{s^2}{2/\sqrt{5}} - \frac{t^2}{2/\sqrt{5}}.
\]
Para $d = 0$, as linhas de nível $\overline{\varphi}(s,t) = d$ são as retas $t = \pm s$.
Para $d \neq 0$, as linhas de nível $\overline{\varphi}(s,t) = d$ são hipérboles.
Nesse caso, temos uma hipérbole com $c^2 = a^2 + b^2$, ou seja, $c = 2\sqrt{d}/\sqrt{\sqrt{5}}$, e portanto os focos da hipérbole são $(-c,0)$ e $(c,0)$, no sistema $s$ e $t$.
Em termos das coordenadas $x$ e $y$, os focos são, respectivamente,
\[
  \left( \frac{-2}{\sqrt{10 \sqrt{5} - 20}}, \frac{4-2\sqrt{5}}{\sqrt{10 \sqrt{5} - 20}} \right), \qquad \left( \frac{2}{\sqrt{10 \sqrt{5} - 20}}, \frac{2\sqrt{5}-4}{\sqrt{10 \sqrt{5} - 20}} \right).
\]

\section*{Seção 23 -- Transformações Lineares}

\textbf{11.}
Uma transformação linear $T : \R^2 \to \R^2$ de posto $2$ transforma toda reta numa reta.
Prove isto.

\vspace{\baselineskip}

\emph{Solução.}
Seja $T(x,y) = (ax + by, cx + dy)$, e seja $M$ a matriz de $T$.
Como $M$ tem posto $2$, os vetores-coluna de $M$ são não-colinares.
Se $r$ é uma reta vertical, então $r$ é formada pelos pontos $(x,y) = (\alpha,t)$ para $t \in \R$.
Logo, os pontos
\[
  T(x,y) = T(\alpha, t) = (a \alpha + bt, c \alpha + dt) = \alpha(a, c) + t(b,d)
\]
para $t \in \R$ formam uma reta, pois $(b,d) \neq (0,0)$ (caso contrário teríamos $ad - bc = 0$, o que é impossível).
Se $r$ é uma reta não-vertical, então $r$ é o conjunto dos pontos $(x,y) = (t, \alpha t + \beta)$ para $t \in \R$.
Logo os pontos
\[
  T(x,y) = T(t, \alpha t + \beta) = \beta(b,d) + t \big( (a,c) + \alpha (b,d) \big)
\]
para $t \in \R$ formam uma reta, pois não existe $\alpha$ tal que $(a,c) + \alpha(b,d) = 0$ (caso contrário $(a,c)$ e $(b,d)$ seriam colineares).

\vspace{\baselineskip}

\textbf{15.}
Seja $T : \R^2 \to \R^2$ uma transformação linear invertível.
Mostre que $T$ transforma retas paralelas em retas paralelas, portanto paralelogramos em paralelogramos.
E losangos?

\vspace{\baselineskip}

\emph{Solução.}
Seja $T(x,y) = (ax + by, cx + dy)$, e seja $M$ a matriz de $T$.
Como $T$ é invertível, para todo $(m,n) \in \R^2$ existe apenas um vetor $(x,y) \in \R^2$ tal que $T(x,y) = (m,n)$.
Dito de outra forma, o sistema de equações
\begin{align*}
  ax + by & = m \\
  cx + dy & = n
\end{align*}
possui apenas uma solução.
Portanto as retas $ax + by = m$ e $cx + dy = n$ são concorrentes.
Logo os vetores $(a,b)$ e $(c,d)$ são não-colineares, ou seja, $ad - bc \neq 0$.
Isso implica que os vetores $(a,c)$ e $(b,d)$ são não-colineares, ou seja, que a matrix de $M$ tem posto 2.
Pela solução do Exercício 11, para qualquer valor de $\alpha$, a transformação $T$ mapeia a reta $x = \alpha$ em uma reta paralela ao vetor $(b,d)$ que passa por $(a,c)$.
Isso mostra que $T$ transforma as retas paralelas $x = \alpha$ e $x = \alpha'$ em retas paralelas ao vetor $(b,d)$.
Pela solução do Exercício 11, a transformação $T$ mapeia a reta $y = \alpha x + \beta$ em uma reta paralela ao vetor $(a,c) + \alpha(b,d)$ que passa por $(a,c)$.
Analogamente, a transformação $T$ mapeia a reta $y = \alpha' x + \beta$ em uma reta paralela ao vetor $(a,c) + \alpha'(b,d)$ que passa por $(a,c)$.
Como $(a,b) + \alpha(c,d)$ e $(a,c) + \alpha'(b,d)$ são vetores colineares, concluímos que $T$ transforma retas não-verticais paralelas em retas não-verticais paralelas.
Além disso, concluímos que $T$ transforma paralelogramos em paralelogramos.
A transformação $T$ não mapeia losangos em losangos, em geral.
De fato, considere o quadrado cujos vértices são os pontos $A = (0,0)$, $B = (1,0)$, $C = (1,1)$ e $D = (0,1)$ (esse é um exemplo de losango).
Observamos que os os vetores unitários $\overrightarrow{AB} = (1,0)$ e $\overrightarrow{AD} = (0,1)$ são mapeados nos vetores $(a,c)$ e $(b,d)$, que não são unitários, em geral.
Logo o quadrado $ABCD$ não é transformado em um quadrado, em geral.

\vspace{\baselineskip}

\textbf{17.}
Dados $u = (1,2)$, $v = (3,4)$, $u' = (5,6)$ e $v' = (7,8)$, ache uma transformação linear $T : \mathbb{R}^2 \to \mathbb{R}^2$ tal que $Tu = u'$ e $Tv = v'$.

\vspace{\baselineskip}

\emph{Solução.}
Uma transformação linear $T : \R^2 \to \R^2$ tem a forma $T(x,y) = (ax + by, cx + dy)$, onde $a, b, c, d \in \R$.
Procuramos constantes $a$, $b$, $c$ e $d$ tais que $T(1,2) = (5,6)$ e $T(3,4) = (7,8)$, ou seja, $(a + 2b, c + 2d) = (5,6)$ e $(3a + 4b, 3c + 4d) = (7,8)$, ou seja, $a + 2b = 5$, $c + 2d = 6$ e $3a + 4b = 7$, $3c + 4d = 8$.
Obtemos portanto um sistema de quatro equações e quatro incógnitas, $a$, $b$, $c$ e $d$.
De fato, obtemos dois sistemas de duas equações e duas incógnitas, desacoplados:
\[
  \begin{aligned}
    a + 2b & = 5 \\
    3a + 4b & = 7
  \end{aligned}
  \qquad \qquad
  \begin{aligned}
    c + 2d & = 6 \\
    3c + 4d & = 8.
  \end{aligned}
\]
Resolvendo esses sistemas, obtemos $a = -3$, $b = 4$, $c = -4$ e $d = 5$.
Portanto, a transformação linear procurada é
\[
  T(x,y) = (-3x + 4y, -4x + 5y).
\]

\section*{Seção 24 -- Coordenadas no Espaço}

\textbf{5.}
Escreva a equação do plano vertical que passa pelos pontos $P\!=\!(2,3,4)$ e $Q = (1,1,758)$.

\vspace{\baselineskip}

\emph{Solução.}
O plano vertical que passa por $P$ e $Q$ deve conter todos os pontos da forma $(2,3,z)$ e $(1,1,z')$ para $z \in \R$ e $z' \in \R$.
Em particular, o plano vertical deve conter os pontos $P' = (2,3,0)$ e $Q' = (1,1,0)$.
Além disso, observamos que o plano vertical deve conter a reta $P'Q'$.
As coordenadas de $P'$ e $Q'$ no plano $\Pi_{xy}$ são $(2,3)$ e $(1,1)$.
Portanto $\overrightarrow{P'Q'} = (-1,-2)$ no plano $\Pi_{xy}$.
O vetor $v = (2,-1)$ é ortogonal a $\overrightarrow{P'Q'}$.
Logo a equação da reta $P'Q'$ no plano $\Pi_{xy}$ é $2x - y = c = 2(1) - 1(1) = 1$, ou seja, $2x - y = 1$.
O plano vertical que passa por $P$ e $Q$ é formado por todos os pontos $(x,y,z)$ tais que $2x - y = 1$.
Essa é a equação do plano.

\vspace{\baselineskip}

\textbf{7.}
Escreva a equação geral de um plano vertical.

\vspace{\baselineskip}

\emph{Solução.}
A equação geral de um plano vertical é $ax + by = c$, onde $a$, $b$ e $c$ são números reais.
De fato, o conjunto de todos os pontos $(x,y,z)$ tais que $ax + by = c$ forma um plano que contém o eixo $OZ$ ou é paralelo ao eixo $OZ$.
(Veja a solução do Exercício 5.)

\section*{Seção 29 -- Equação do Plano}

\textbf{2.}
Obtenha uma equação para o plano que contém $P$ e é perpendicular ao segmento de reta $AB$ nos seguintes casos:
\begin{enumerate}[(a)]
  \item
    $P = (0,0,0)$, $A = (1,2,3)$ e $B = (2,-1,2)$.
  \item
    $P = (1,1,-2)$, $A = (3,5,2)$ e $B = (7,1,12)$.
  \item
    $P = (3,3,3)$, $A = (2,2,2)$ e $B = (4,4,4)$.
  \item
    $P = (x_0, y_0, z_0)$, $A = (x_1,y_1,z_1)$ e $B = (x_2,y_2,z_2)$.
\end{enumerate}

\vspace{\baselineskip}

\emph{Solução.}
(a)
Observamos que o plano é perpendicular à reta $AB$ se e somente se o plano é perpendicular à reta $OB'$ com $B' = (2-1,-1-2,2-3) = (1,-3,-1)$.
Logo uma equação para o plano é $x - 3y - z = d$ para alguma constante $d$.
Como $P$ pertence ao plano, devemos ter $1(0) - 3(0) -1(0) = d$, ou seja, $d=0$.
Portanto uma equação do plano é $x - 3y - z = 0$.

(b)
Como fizemos no item (a), chegamos à equação $4x - 4y + 10z = -20$.

(c)
Como fizemos no item (a), chegamos à equação $2x + 2y + 2z = d$.

(d)
Como fizemos no item (a), chegamos à equação $(x_2 - x_1)x + (y_2 - y_1)y + (z_2 - z_1)z = (x_2 - x_1)x_0 + (y_2 - y_1)y_0 + (z_2 - z_1)z_0$.

\end{document}
